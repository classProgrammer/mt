The 3 Banken IT wants to use chatbots for two internal use-cases.
One use-case is about sickness notifications, and the second one is about vacation requests.
The required key feature is the extraction of information from text since both use-cases require the extraction of person names and dates to be automated.

Before the development phase of a chatbot begins, the given problem needs to be evaluated. 
A problem is suitable for a chatbot if it is repetitive, can be automated, and only requires simple back-and-forth communication.
The given use-cases are suitable for a chatbot since all three criteria are satisfied. 

Prototypes for the use-cases were developed with Dialogflow, LUIS, Rasa, and Watson Assistant to validate the suitability of the use-cases and to do the proof of concept.
The technologies were ranked using factors like the entity extraction performance, the intent classification performance in English and German test scenarios, the usability of the frameworks, the customization possibilities, and learnability.
Contrary to most sources found on the internet, the evaluation included development relevant aspects of the frameworks to give a development recommendation to the company. 

A key factor why companies want to use chatbots is because they are cheap and always available.
The evaluation includes the price comparison of the frameworks, text-to-speech, and speech-to-text services.
The costs for the cloud chatbots and speech services were on an equal level as expected.
Rasa does not list the costs for the enterprise edition on the website, nor does Rasa offer text-to-speech or speech-to-text services.

The comparison of the frameworks' similarities and differences led to the discovery of common and technology-specific concepts and development-relevant evaluation criteria.
All four technologies use intents, entities, and utterances.
They provide a dialog handling mechanism.
The most significant difference is that Dialogflow, LUIS, and Watson Assistant are cloud-based solutions, whereas Rasa is a local solution.
The local solution Rasa offers advantages when it comes to offline capabilities, flexible deployment, training, and independence of providers.
The cloud technologies were, in general, easier to use and learn, the provider handles the deployment, they scale automatically and perform better on average with small training sets. 

The two primary functions of chatbots are entity-extraction and intent classification.
Dialogflow won the entity extraction evaluation by a large margin backed up by the highest f-score.
The intent classification in English worked best with Watson Assistant, closely followed by Dialogflow and LUIS.
Rasa was unable to keep up with the cloud technologies in the English test setting.

The results were quite different for the German test sentences.
The intent classification worked measurably better with all four technologies.
The entity extraction performance dropped significantly for the three chatbot technologies, while the performance of LUIS increased.
The only technology which showed excellent performance on the German tests was LUIS, followed by Rasa, which performed quite well.
The entity extraction performance of Dialogflow on the German tests was horrible.
The recommended technologies for further projects are LUIS and Rasa.
