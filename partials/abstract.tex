The 3 Banken IT wants to use chatbots for two internal use-cases.
The first use-cases are sickness notifications and the second one 
are vacation requests.
The main questions which need to be answered are 
what makes a problem suitable for a chatbot solution and are the given 
problems suitable for a chatbot.
A problem is suitable if it's repetitive, can be automated, and 
only requires simple back-and-forth communication \citet{buiildChatbotsPython}.
This is the case for the given problem which makes it suitable for a
chatbot solution.
For the proof of concept, prototypes were developed with Dialogflow, LUIS, Rasa, 
and Watson Assistant.
Based on a detailed evaluation that included the entity extraction performance, 
the intent classification performance in English and German a technology ranking 
was built to determine the best technology for the given problem.
In the evaluation process the similarities and differences of the 
frameworks were compared.
All of the tested frameworks used the concepts intent, entity, and utterance.
The biggest difference is that Dialogflow, LUIS, and Watson Assistant are 
cloud-based solutions whereas Rasa is a local solution.
The evaluation also focused on the difference between the cloud frameworks and the 
local framework.
An important factor for companies are the costs for a technology and in general the 
prices of the frameworks are on an equal level as expected.
The entity extraction evaluation was won by Dialogflow by a large margin backed up 
by the highest f-score.
The intent classification in English worked best with Watson Assistant closely followed 
by Dialogflow and LUIS.
Rasa was unable to keep up with the cloud technologies in the English test setting.
The result for the German test setting looked quite different since the 
intent classification worked measurably better and all technologies 
performed well.
The entity extraction performance of the technology dropped significantly with the 
exception of LUIS which showed a performance increase.
For the development of a prototype in English, all technologies can be recommended.
The only technology which showed a great performance on the 
German tests was LUIS followed by Rasa which performed quite well.
These two technologies can be recommended for a German bot.
The entity extraction performance of Dialogflow was horrible and can't be
recommended.
Watson Assistant doesn't offer annotated entities for training sentences when German 
is selected as input language.
This renders Watson Assistant useless for the German language until the feature is 
available for German and can't be recommended.
The two remaining technologies for the final recommendation are LUIS followed by Rasa.  