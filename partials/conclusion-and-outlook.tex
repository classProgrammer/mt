The domain consisting of the sickness notification and the vacation request 
can be solved with chatbots.
% implementation
It was possible to implement the design with all three chatbot 
technologies namely Dialogflow, Rasa, and Watson Assistant.
Based on Chapter \ref{chap:proto} the implementation is 
easiest with Dialogflow then comes Watson Assistant followed by 
Rasa which is the most complicated by far.
It's easiest with Dialogflow because it already has prebuilt entities,
it offers the utility intents yes and no, and small talk can be 
included with a single mouse click.
Watson Assistant is second because it takes more time to get accustomed 
to the UI, the person entity needs to be defined by the developer,
and the yes/no logic needs to be implemented.
Rasa is the most complex one since it doesn't provide any entities 
itself.
To enable the entities of other technologies like Spacy and Duckling the 
pipeline needs to be adjusted.
To connect to Duckling and Additional Docker container needs to be started.
The chatbot developer also needs to implement the form filling with Python 
when using Rasa. 
The other technologies don't require any programming skills to model the 
dialog.
The implementation with Rasa took by far the longest.
When only the implementation section is used for the ranking 
Dialogflow wins in front of Watson followed by Rasa.
% no of entities
Table \ref{tab:predefined_entities} shows the number of predefined entities.
When only the domain of this thesis is considered then the only technology 
where an entity is missing would be Watson Assistant since it provides no person entity.
When the number of entities is considered for a general case then the 
top technology by a large margin is Dialogflow.
The number of predefined entities is an important number for future use-cases since 
lots of development time can be saved when the entities are already predefined.
They are also trained well because they use large training sets.
% entity extraction
Table \ref{tab:entity_extraction_eval} ranks the frameworks based on their 
entity extraction capabilities.
The best person, date and date-span entities are present at Dialogflow because they achieved the 
highest f-score values.
In the date-span entity category, Duckling and Watson achieved the same f-score as Dialogflow.
The best overall entity extraction performance is achieved by Dialogflow. 
From the entity extraction capabilities alone either Rasa or 
Dialogflow needs to be recommended.
The other frameworks performed worse in the entity extraction 
department.
The ranking shows that some technologies performed better than others.
It also shows that no framework performed badly.
% intent classification
The intent classification is dominated by Watson and Dialogflow since 
both were able to classify all test cases correctly.
LUIS is a close second.
All three cloud chatbots performed on an equal level.
The detailed result is shown in Table \ref{tab:intent_classification_result}.
Rasa was unable to keep up with Dialogflow, LUIS, and Watson when the 
average confidence is compared.
This is an indication that Rasa needs more training data to 
reach the same performance level as Dialogflow, LUIS, and Watson.
An average confidence value below 50\% is not satisfactory.
Based on the entity extraction and intent recognition results
the recommended technology in the top spot has to be Dialogflow
since the performance in both categories was great.
Additionally, Dialogflow has the best average confidence value.
It's also possible to use LUIS and Watson Assistant since the performance was close to Dialogflow.
It's possible to improve the performance and confidence scores of Rasa with more training data.
Rasa needs to improve to be recommended.
% intent classification german 
The intent classification was also done for the German language and the results differ greatly from the 
English results.
The average confidence increased for each technology.
The f-score of LUIS and Rasa increased significantly. 
This puts LUIS on par with Dialogflow and Watson with a perfect f-score value of 1.0.
This leads to the conclusion that the intent classification works better for the German 
language in general since all technologies performed better.
Contrary to the intent classification, the entity-extraction performance decreased significantly.
In the English test sentences of Tables \ref{tab:sickness_intent_classification} and \ref{tab:vacation_intent_classification}
the entity extraction didn't work correctly in two cases and worked in 46 cases.
For the German language, test shown in Tables \ref{tab:sickness_intent_classification_de} and \ref{tab:vacation_intent_classification_de},
a total of 21 wrong and 26 correct extractions are counted.
The only technology that achieved a satisfying entity extraction result on the German test sentences is LUIS.
% portability
Rasa is the only portable technology.
It can run in a Docker container and Docker containers are supported by every 
cloud provider.
It can also run locally on a development machine and on an internal server.
The cloud technologies are tied to the cloud provider who offers the bot.
Rasa wins the portability comparison.
% migration
The migration from one chatbot technology to another is not easily possible with 
any of the technologies tested. 
Hence, there is no evaluation possible and no ranking can be done.
% pricing
The pricings of the frameworks shown in Table \ref{tab:pricing} are hard to compare and intransparent for the premium version of Watson Assistant and the enterprise edition for Rasa.
All three offer a free tier which can be used by companies.
Because of the lack of transparency, the comparison can't be used for a recommendation or a ranking.
The real difference is that Watson Assistant is measured by the number of users per month and 
Dialogflow and LUIS + Microsoft Bot framework are based on the number of requests.
This makes Dialogflow and LUIS cheaper for small amounts of requests and lots of users with short conversations.
Watson is cheaper for returning users with lots of conversation or very long conversations.
The prices for the speech capabilities of Dialogflow and the Microsoft Speech Services are comparable.
The same functionality is more expensive when the services of IBM are used.
Worst in this category is Rasa since it offers no speech services but it's possible to use the 
speech services of the cloud providers.
The winners of the speech capability evaluation are Dialogflow and the Microsoft Speech Services.
% security
When security is an important factor then a local technology needs to be used.
The cloud technologies all process the requests in the cloud and this has to be 
kept in mind.
If it's impossible to process information in the cloud because there are security restrictions
then the recommendation is to use Rasa.
% recommendation of a framework
When all the factors discussed above are combined the clear winner for English bots is Dialogflow and the 
winner for German bots is LUIS.
The design could be implemented fastest with Dialogflow, it's the easiest of the chatbot technologies,
it offers the most predefined entities, provides utility intents, had the best person and date
entity in the entity extraction tests has the highest average confidence score, 
and performed great in the intent classification tests.
The only downside of Dialogflow is the entity extraction performance on the German test sentences.
The best overall technology is LUIS and is the recommended framework for future development.
The only negative aspect of Rasa is the low average confidence score but it can be 
improved with more training data and is a good local alternative to LUIS.
% differences and similarities
When looking at the differences and similarities of the chatbot framework the conclusion is that 
they are very similar and differ only in small aspects.
Especially the cloud frameworks are almost identical in functionality, the provided services,
and the costs.
All tested frameworks use the concepts intent, entity, and utterance.
The chatbot frameworks use a mechanism to structure the dialog with the user.
Dialogflow and Watson Assistant use dialog nodes and Rasa uses stories.
The development effort and complexity between the technologies vary.
Cloud chatbots require less knowledge and are easier to use while the local technology (Rasa) 
offers more customization possibilities.
The pipeline of Rasa can be adjusted to the specific needs of the developer or company which 
is not offered by any cloud technology.
% suitable problems
The third research question is about the suitability of problems for the use of chatbots.
The three major criteria mentioned by \citet{singhbuilding} can be confirmed.
A problem needs to be solvable via simple back and forth communication because it's the 
only thing a chatbot can do.
The problem is not required to be repetitive but it's only cost-efficient if it is 
a repetitive problem.
It should also be automatable to be efficient and to make the chatbot work independent.
In general, all problems which satisfy those three criteria are suitable for the use of chatbots.
Possible future use cases for the 3 Banken IT are the use of a chatbot for a banking
application like mentioned in \citet{singhbuilding}.
% further use cases
The banking app of \cite{singhbuilding} is used for transactions and querying account information.
A popular use case for chatbots are FAQs(\citet{evaluateChatbotsShawar2007, buiildChatbotsPython, huang2007extracting, GO2019304}).
% advice for future work
When resources like \citet{} and \citet{} are taken into account the psychological aspects of 
chatbot development have to be considered for future work.
Like \citet{GO2019304} said, there are multiple ways to make users believe that they are talking to a bot.
For an actual chatbot, there needs to be decided if the users should know that they are talking to a bot or not.
If they believe to talk to a bot they expect less but the conversation is generally treated more machine-like \cite{GO2019304}.
When users assume to talk to a human the conversation is judged as more natural but if the bot performs badly the experience is also evaluated as more negative \cite{GO2019304}.
Like \citet{shawar2007chatbots} said, a chatbot should not imitate humans they should be used to reduce their workload and help them.
% source of data and user study
For prototyping the generation of training data by the developer is appropriate.
It's not appropriate for the training of the actual product since the performance
requirements are much higher.
The training data for the future work on the actual product can be collected through various
real-life conversation sources like emails and conversations between customers and experts.
With real-life data, the training data is as realistic as possible and the quality of the chatbot 
increases.
% user-study
User-studies are recommended to increase the quality of a chatbot further since the feedback 
of actual users is the most valuable to improve the actual systems.