\section{Architecture}
\makefigure{simplearchitecture}{Example Minimalistic Architecture}

One possible architecture for a minimalistic chatbot system with database communication is shown in Figure \ref{fig:simplearchitecture}.
The chatbot framework takes care of all conversation related tasks. 
Conversation related tasks are the intent recognition, entity extraction and the general flow of the dialogue for instance. 
The user interface of Figure \ref{fig:simplearchitecture} fetches the input from the user, submits it to the chatbot framework 
and displays the response messages. 
The UI can additionally communicate with the database via the REST service or webhook if necessary.
The DB handles data-related tasks like the possible items and can be used for the validation of the extracted entities
 by the chatbot framework in combination with the webhook. 
 The webhook separates the data logic from the application logic and is used for validation, 
 data submission and retrieval.
 It's also used for other chatbot related issues like the validation of intent related data and error responses to the chatbot framework.

 \section{Story/Flowchart}
 
 \makefigure{sicknessflow}{Example Conversation for a Sickness Notification}
 Like mentioned in \citet{folstad2017chatbots} the design focus for chatbot systems lies on the conversation itself.
 A story is an example dialog/conversation between user and chatbot. 
 An example dialogue to report a colleague sick is shown in Figure \ref{fig:sicknessflow} 
 where the user input is highlighted green and the chatbot response is highlighted blue.
 The intent of each user input needs to be found to create the knowledge base for the domain. 
 The intent for the user input "Hello" can be marked as greeting, the second input is the general information what the chatbot can do, 
 the third input is the sickness notification intent. Then the necessary information is collected followed by some chit-chat and the end of the conversation. 
 The interesting entities, in this case, are the name of the employee which is required all the time, the confirmation that the found employee is the 
 correct one and the return date which is optional since it's not possible to tell in every case. 
 The resulting intents, entities, and utterances for the example sentences of the conversation 
 shown in Figure \ref{fig:sicknessflow} are listed in Table \ref{tab:sick_data}.
 
 
 
 \section{Intents, Entities and Utterances}
 The base for this section is Figure \ref{fig:sicknessflow} which shows an example conversation 
 for a sickness notification from start to end.
 The identified information from the conversation is displayed in Table \ref{tab:sick_data}.
 In genral, the greeting and goodbye intents are expected to be in almost any chatbot conversation.
 A thank you intent is also very likely.
 The user should have the option to ask for the available options.
 The core intent of the whole process is the sickness intent.
 To do the sickness notification the users needs to enter the name of a person.
 If the person can't be found the user needs to be notified and the operation is aborted.
 If the person was found the user needs to confirm that the right person was found.
 In production there could be multiple employees with the same name. 
 Because of this ambiguity the confirmation is part of the dialog.
 Then the optional return date needs to be extracted.
 It's optional because especially in the case of a sickness the return date is unknown.
 The other case is e.g. a broken arm where the employee might return tomorrow.
 Then the sickness notification is done.
 In the case of this thesis the chatbot is the important part hence the backend is mocked for simplicity.
 The user might thank the bot or end the conversation with a goodbye message, or the user might not respond after the all done message.
 This are the three possible ends of conversation.
 The basic concepts extracted from Figure \ref{fig:sicknessflow} are shown in Table \ref{tab:sick_data}.
 There are the intents discussed above, the entities and their types, and the bot actions.
 For this use case the needed entities are a person or the name of a person, a confirmation of boolean type, and a data.
 Alternatively for the confirmation the name of the department the person works in can be used.
 The department is modeled as a custom entity as described in Chapter \ref{chap:basics}: Basics in Section \ref{sec:entity}.
 The bot responds to the user with utterances.
 The conceptual names of the utterances are show in Table \ref{tab:sick_data} under Bot Action.
 In the simplest casa a bot action is an utterance.
 For the utterances different version need to be written to respond to the user with alternating sentences.
 For the vacation notification the process remains the same.
 The greeting, thanks, goodbye, and list options intents remain the same.
 The new intent is the vacation intent.
 To validate a vacation request the name of the person and the department are needed again.
 The only change in the process is that there needs to be a mandatory start and end date.
 There are no new datatypes/entities needed for the second use-case.

 \begin{table}[h]
    \centering
    \begin{tabular}{ l || l | l | l }
        Sentence & Intent & Entities & Bot Action \\ \hline \hline
        Hello & Greet & - & Utter Greeting \\ \hline
        What can you do? & List Options & - & Utter Options \\ \hline
        my colleague is sick & Sick & - & Utter Ask Name \\ \hline
        Max Power & - & Person Name & Utter Confirm \\ \hline
        Yes & - & Boolean & Utter Return Date or Failed \\ \hline
        No & - & Date or None & Utter Done or Failed \\ \hline
        Thanks & Thanks & - & Utter You'r Welcome \\ \hline
        Bye & Goodbye & - & Utter Goodbye \\ \hline
    \end{tabular}
    \caption{Sickness Norification Intents, Enitities, and Actions of Figure \ref{fig:sicknessflow}} \label{tab:sick_data}
\end{table} \noindent

\section{Problem Analysis}
The problem is task-oriented.

Some psychological aspects of \citet{brandtzaeg2018chatbots} have to be included.
\citet{brandtzaeg2018chatbots} stated that the user impression depends greatly on the users mindeset.
If the user thinks he's talking to a bot the user expects less but the conversation is more likely 
evaluated as unnatural \cite{brandtzaeg2018chatbots}. If the user asumes he's talking to a human 
and the bot is unable to achieve the taks the user is very disappointed \cite{brandtzaeg2018chatbots}.
Since the problem is a task-oriented one the general conversation capabilities are not the primary focus.
This means the bot is unable to fake a human conversation which goes beond the expected functionality.
Hence, the user needs to be clearly informed that hes talking to a bot.
For further development steps a gender can be introduced to make the bot feel more human like if necessary.
Another way to give the bot a more natural feel is to look at the message response times.
They should not be to fast.
If the response comes too fast it feels unnatural because a human being can't answer as fast as a machine.

\section{Webhook}
As explained in Chapter \ref{chap:basics}: Basic in Section \ref{sec:webhook} 
the webhook is an endpoint where services can send requests to.
In this thesis three frameworks are present and each framework gets and endpoint.
The service is also used to access general information about the state of the data.
The Dialogflow endpoint is reachable under "/dialogflow/webhook",
the Watson endpoint is located at "/watson/webhook",
and the Rasa service submits post requests to "/sick" and "/vacation" to submit data 
to the REST service.
For development purposes the last request of Watson can be viewed at "/watson/request",
and the last request of Dialogflow at "/dialogflow/request".
To see all successful sickness notification entries "/sick/all" can be used,
and under "/vacation/all" all successful vacation entries can be viewed.

The service fakes a database and stores the request.
After the information has been collected successfully by a framework a 
request is submitted to the service.
The service can do everything with the data and is the point where the company can execute actions.
All frameworks can call the webhook at the end when the data has been collected but also 
after each user request if necessary.
This way the information can be validated and modified by the service.



\section{Messenger UI}

