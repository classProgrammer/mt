% price
The price comparison of Table \ref{tab:pricing} shows that all 
tested frameworks can be used for free.
The non-free tiers of Dialogflow and LUIS in combination with the Microsoft Bot framework 
are charged based on the number of requests while the price for Watson Assistant
is calculated based on the number of users.
The price for the enterprise edition of Rasa and the premium edition of Watson Assistant 
is not stated on the websites.
The prices are hard to compare but all are offered in a free tier and the 
bots of Google and Microsoft cost roughly the same per request. 
The disadvantage of Rasa is the deployment since the costs for the infrastructure are unclear 
for a similar workload.
Since Watson Assistant calculated the price based on the number of users it offers a better 
value when the users communicate with the bot frequently or have long conversations.
The prices for the speech functionality are comparable for Dialogflow and the Microsoft speech-services.
The IBM text-to-speech service is more expensive than the one of Dialogflow and Microsoft.
Rasa doesn't offer any speech services.
This means that the Microsoft speech-service and Dialogflow are the best when it comes to the 
price. Watson is second and Rasa doesn't show any prices at all.
% where can it run
A big disadvantage of cloud-based solutions is the dependency on the company.
It's not easy to change the provider and take the current chatbot with you if it 
is a cloud-based chatbot. 
Rasa differs at that point since it can run in any cloud environment inside a docker container.
Every cloud provider offers the option to host docker containers in their cloud.
A major factor is security since the users enter sensitive information in the chatbot communication
and the data is processed in the cloud of the provider.
The cloud provider hence has access to the sensitive information which can be problematic with
bank account information for instance.
Rasa has no such problem since the data is processed inside of the docker container or in a 
local environment.
% entity extraction
The entity extraction section shows that Dialogflow and LUIS 
define the required entities while Watson lacks a person entity.
Rasa doesn't offer any entities but can use the entities of 
Spacy and Duckling.
If both are used then no entities need to be defined in Rasa.
The result in Table \ref{tab:entity_extraction_eval} shows that 
Dialogflow has the best person, date, and date-span entity and dominated the 
entity extraction tests.
The best overall performance was achieved by Dialogflow followed by Rasa.
No technology performed exceptionally bad in this category and the date-span entity of 
Duckling, Dialogflow, and Watson Assistant reached the same f-score which means they 
performed on an equal level.
% intent classification
The intent classification result of Table \ref{tab:intent_classification_result}
shows that the cloud services performed on an equal level when it comes to 
precision, recall, the f-score, and the average confidence.
Rasa was unable to keep up with the cloud technologies and especially the average 
confidence score is low.
Watson Assistant and Dialogflow were able to reach a perfect f-score of 1.0.
The standard deviation is lowest for Watson which puts Watson Assistant in the top 
spot closely followed by Dialogflow and LUIS.
Dialogflow extracted an entity incorrectly 
in test case four of Table \ref{tab:sickness_intent_classification}.
Rasa showed a good performance when the f-score is used as a measure but 
has a low average confidence score of 46\%.
The confidence scores of Dialogflow, LUIS, and Watson are far better with 77\%, 75\%, and 74\%.
In this category, Dialogflow, LUIS, and Watson performed good enough to be recommended.
The confidence scores of Rasa can be improved through more training data.
% german intent classification
The german language result of Table \ref{tab:intent_classification_result_de} shows some interesting 
differenced to the English tests of Table \ref{tab:intent_classification_result}.
The average confidence increased for all four technologies and the standard deviation 
decreased significantly for four technologies.
For the German language, the f-score values also increased.
All in all the German language result looks very good on the surface when only the intent 
classification is considered.
The entity extraction performance dropped significantly on the test sentences.
The German tests are shown in the appendix in Table \ref{tab:sickness_intent_classification_de} and \ref{tab:vacation_intent_classification_de}.
There were a total of 21 failed extractions for the German test sentences and a total of three failed 
extractions for the English test sentences on a total of 64 test sentences each.
This means in German the entity extraction failed on every third test sentence.
The only technology which delivered a satisfying entity extraction result on the German 
test sentences with zero mistakes total is LUIS.
The performance of Rasa was worse than the performance of LUIS but it was good enough
for a recommendation.
The date entity of Dialogflow and Watson Assistant are unable to cope with the German date format.
The person entity of Dialogflow did not work reliably in German although the names aren't 
different from the ones used in the English test sentences.
% sparse data same performance
\citet{braunEvaluatingNLU} stated that if the training data is sparse there is almost no 
performance difference between the chatbot frameworks regarding NLU.
This can be confirmed for the cloud frameworks since there is no big difference reagarding the f-score, 
average confidence, and entity extraction, shown in in Tables \ref{tab:entity_extraction_eval}, \ref{tab:intent_classification_result}, \ref{tab:intent_classification_result_de}.
The local solution Rasa was measurably worse.
% communication
The communication of the frameworks works the same way for all the technologies.
There is no big difference no matter what is used.
LUIS and Dialogflow offer metadata by default while Rasa and Watson Assistant don't.
All four offer REST endpoints and use JSON format for messages.
% Learnability
The easiest technology is LUIS because it's just an NLU framework and doesn't handle the dialog structure.
Dialogflow is the easiest chatbot framework because it has the simplest UI, the most predefined entities,
predefined intents for common tasks like a yes/no path and small talk can be enabled in the menu which 
saves development time.
Watson is comparable to Dialogflow when it comes to the creation of entities and intents but the 
UI is less intuitive, there is no small talk feature, there are almost no predefined entities, 
and offer no predefined intents for common tasks.
The most complex is Rasa simply for the facts that the base version has no UI and programming 
skills are required.
To add predefined entities the pipeline needs to be adjusted, it has no predefined intents, 
and a small talk feature is not available.
The creation of dialogs with Rasa is far more complicated as is the collection of form data.
Compared to the cloud frameworks Rasa is incredibly complicated and doesn't offer a lot of 
features by default.
% setup
The setup of Rasa is complicated compared to the cloud chatbots which was expected.
To use a cloud chatbot only a browser is needed, for Rasa either Docker is required or a Python
environment.
It's harder to set up Rasa because the cloud technologies run already in the cloud and require 
no additional deployment.
The comparably complex setup has advantages when it comes to deployment.
% deployment 
Rasa can be deployed locally and in every cloud environment while the cloud services only 
run in the cloud of the provider.
The deployment evaluation met the expectations.
Rasa is flexible and the cloud chatbots are simple when it comes to deployment.
% portability
Since Rasa can be deployed anywhere, doesn't rely on the cloud, and can run without an
internet connection it's the only real choice of the four for security-critical applications.
% implementation
Rasa is more complicated but it was possible to implement the design with all chatbot technologies.
This means that all three technologies can be recommended when only the implementation of the design
is considered.
% concepts
The basic concepts for the implementation are the same for all technologies, there are intents,
entities, and utterances.
Dialogflow and Watson Assistant use dialog nodes for the conversation structure while Rasa 
uses stories. 
But in the end, there is no big difference when it comes to the concepts.
% customization
The only technology which allows customization is Rasa.
If something very specific is needed or tuning is required then Rasa should be used.

