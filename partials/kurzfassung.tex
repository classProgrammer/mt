Die 3 Banken IT möchte Chatbots für zwei interne Anwendungsfälle verwenden.
In einem Anwendungsfall geht es um Krankheitsmeldungen, im zweiten um Urlaubsanfragen.
Eine für die Anwendungdfälle notwendige Funktion ist das Extrahieren von Informationen aus Text, da in beiden Anwendungsfällen das Extrahieren von Personennamen und Datumswerten automatisiert werden muss.

Bevor die Entwicklungsphase eines Chatbots beginnt muss das Problem evaluiert werden.
Ein Problem ist geeignet, wenn es sich um eine sich wiederholende Aufgabe handelt, automatisiert werden kann und nur einfache Bidirektionale Kommunikation im Frage Antwort Schema erforderlich ist.
Die gegebenen Anwendungsfälle erfüllen alle drei Anforgerungen und sind daher geeignet.

Mit Dialogflow, LUIS, Rasa und Watson Assistant wurden Prototypen für die Anwendungsfälle entwickelt, um die Eignung der Anwendungsfälle zu validieren und den Proof-of-Concept durchzuführen.
Die Technologien wurden anhand von Faktoren wie der Entitätsextraktionsleistung, der Absichtsklassifizierungsleistung in englischen und deutschen Testszenarien, der 
Benutzerfreundlichkeit, den Anpassungsmöglichkeiten und der Erlernbarkeit eingestuft.
Im Gegensatz zu den meisten im Internet gefundenen Quellen umfasste die Bewertung entwicklungsrelevante Aspekte der Frameworks, um dem Unternehmen eine Entwicklungsempfehlung zu geben.

Unternehmen verwenden Chatbots weil sie billig und immer erreichbar sind.
Die Kosten für die Cloud-Chatbots und der Text-zu-Sprache und Sprache-zu-Text Dienste liegen ungefähr auf dem gleichen Niveau.
Alle Cloud-Anbieter bieten Sprachdienste an während Rasa weder die Kosten für die Enterprise-Edition auf der Website auflistet, noch Text-zu-Sprache und Sprache-zu-Text Dienste anbietet.

Der Vergleich der Gemeinsamkeiten und Unterschiede der Frameworks hat allgemeine und technologie-spezifische Konzepte und entwicklungsrelevante Bewertungskriterien ans Licht gebracht.
Alle vier Technologien verwenden Absichten, Entitäten und Äußerungen zur Darstellung und Speicherun von Informationen und verfügen über einen Dialogbehandlungsmechanismus.
Der markanteste Unterschied ist, dass Dialogflow, LUIS und Watson Assistant Cloud-basierte Lösungen sind, während Rasa eine lokale Lösung ist.
Rasa bietet Vorteile in Bezug auf Offline-Funktionalität, flexiblem Deployment, dem Trainieren des Chatbots und ist Unabhängig von Anbietern.
Die Cloud-Technologien sind einfacher zu verwenden und zu erlernen, der Anbieter übernimmt das Deployment, sie skalieren automatisch und erzielen mit kleinen Trainingssätzen bessere Ergebnisse.

Die beiden Hauptfunktionen von Chatbots sind die Entitätsextraktion und die Absichtsklassifizierung.
Dialogflow gewann die Bewertung der Entitätsextraktion mit großem Vorsprung, und erreichte den höchsten f-Score.
Die Absichtsklassifizierung in Englisch funktionierte am besten mit Watson Assistant, dicht gefolgt von Dialogflow und LUIS.
Rasa konnte mit den Leistungen der Cloud-Technologien nicht Schritt halten.

Die Ergebnisse für die deutschen Testsätze zeigen ein anderes Bild.
Die Absichtsklassifizierung funktionierte mit allen vier Technologien messbar besser während die Leistung der Entitätsextraktion bei allen drei Chatbot-Technologien erheblich zurückging.
LUIS war die einzige Technologie, die bei den deutschen Tests eine hervorragende Leistung zeigte, gefolgt von Rasa mit einer ordentlichen Leistung.
Die Entitätsextraktionsleistung von Dialogflow bei den deutschen Tests war schrecklich.
Für weitere Projekte werden daher die Technologien LUIS und Rasa empfohlen.