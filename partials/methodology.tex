
\section{Approach}
% was wird betrachted, was nicht        
% Domain wird betrachtet
% andere Sachen nicht
% gleiche test sätze und entitäten
% gleiche stories
The problem (sickness notification) needs to be evaluated to determine if the problem is suitable for chatbots.
Sickness notifications are a repetitive task, the task can be automated and requires simple Back-and-Forth communication hence the problem fits the description of \citet{buiildChatbotsPython}.
The same is true for holiday notifications since the idea of both remains the same.
The problem is suitable for chatbots since all three questions can be answered with yes.
The next step is to determined the converstaion type.
The two major chatbot conversation categories mentioned in Chapter \ref{chap:soa}: State-of-the-Art are goal-oriented 
and end-to-end conversations \cite{williams2017hybrid, bordes2016learning, rahman2017programming}.
A goal-oriented system helps a user to achieve a task \cite{rahman2017programming}.
In the case of a sicknes notification the system needs to aid and guide the user through the process to reduce the workload.
This description fits perfectly for goal oriented dialog.
The domain is clear and of limited size since the user can only ask about the sickness notification process and nothing else.
The bot also needs to retrieve specific data from the conversation to identify the sick person.
The retrieval process is a specific task with specific entities and fits a goal-oriented system.
In open-ended systems the conversation is not limited to a domain and the conversation can evolve in any direction.
This is definitely not the case for the sickness and holiday notifications because the problems are domain specific.
Therefore, goal-oriented technologies fit the the problem description.
End-to-end systems won't be considered in this thesis because their performance in goal-oriented settings is not good enough yet as 
\citet{bordes2016learning} mentioned.
The given problem of sickness notifications is a domain-specific/goal-oriented 
(\citet{deshpande2017survey, luis2015williams, braunEvaluatingNLU, williams2017hybrid}) 
task hence domain-specific frameworks are suitable.

The framework selection for this thesis is based on chapter \ref{chap:soa} State-of-the-Art and Section \ref{sec:prereq} Prerequisites.
Section \ref{sec:prereq} describe the requirements of the company and chapter \ref{chap:soa} shows which technologies are 
used in papers, articles, and books.
To match the prerequisites at least one cloud (\citet{braunEvaluatingNLU, rahman2017programming}) and one local (\citet{braunEvaluatingNLU}) chatbot are needed and IBM Watson Assistant needs to be taken because an IBM technology is on the wish-list.
A common cloud technology is Dialogflow (\citet{braunEvaluatingNLU, dutta2017developing, singhbuilding, buiildChatbotsPython, rahman2017programming, ieee2018watson}) hence it is selected.
A common local standalone technology is Rasa (\citet{braunEvaluatingNLU, singhbuilding, rasabocklisch2017, buiildChatbotsPython, gregori2017evaluation}) and is the local tool of choice.
The chosen technologies are Dialogflow, IBM Watson (\citet{rahman2017programming, pharmacybot, ieee2018watson, gregori2017evaluation}), and Rasa to fulfill all requirements.

All three prototypes will be used to implement the same task.
An example convsersation of the sickness notification task is shown in Figure \ref{fig:sicknessflow}
They will be trained with the same data.
They will have the same intents, entities, and utterances.
These steps are necessary to ensure a fair comparison.
The analysis of the sickness notification dialog is shown in Table \ref{tab:sick_data}.


\section{Model}


\section{Course of Action} % Ablauf
% Domain
% Knowledge Base bilden
% Intents
% entities
% utterances
% test set bilden
% TP
% TN
% FP
% FN
% Stories bauen
% Stories umsetzen
% testen
% deployen für gleiche Bedingungen
% testen und vergleichen