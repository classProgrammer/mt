\section{Dialogflow}
To be done \cite{dialogflow}.

\section{Watson Assistant}
To be done \cite{watsonassistant}.
 
\section{Rasa}
To be done \cite{rasa}.

\section{LUIS}
With LUIS\cite{luisdocs}, the language understanding service of Microsoft,
NLU services can be created.
NLU is only a part of a chatbot and it's not possible 
to create a chatbot with LUIS alone.
The other frameworks are capable to create a chatbot application.
The chatbot features are availiable through the Microsoft bot services 
in combination with LUIS.
A LUIS account is necessary to create NLU services.
When a new project is create a name and a language need to be defined for 
the NLU application.
The choosen language is the language the users will us for the communication with the 
LUIS application.
The modification is done in the build tab.
The interface shown in Figure \ref{fig:luis_interface} offers intents and entities
like described in Chapter \ref{chap:basics}.
To create the NLU part with LUIS Table \ref{tab:conversation_data} will be used.
\ref{tab:conversation_data} defines the intents entities and response utterances
which need to be created in LUIS.
Figure \ref{fig:conversationflow} has no influence on LUIS because the 
conversation handling is not part of the NLU service.
To create an intent only a name and a list of traing phrases (utterances) are necessary.
Multiple options are available to create different kinds of entities.
There are simple entities like e.g. a city.
They describe exactly one concept.
There are composite entities which are built from multiple entities (parts).
E.g. a ticket order entity could consist of a order number, the number of 
tickets, and the event.
An entity list is used when the valid items are limited like in the book hotel example.
There is only a limited number of valid hotel names and the list holds all valid 
hotels.
The invalid hotels are uninteresting in such a case.
Regular expressions\cite{regex} can also be used to define an entity.
It is also possible to add prebuilt entities and prebuilt domain entites.
The entities required for the sickness intent are person name and date. 
LUIS offers the prebuilt entities date-time and person-name.
This means no new entities need to be created for the sickness use-case
since the are prebuilt.
To test the NLU functionality two additional intents need to be created to 
simulate the conversation.
A submit name and a submit return date intent are required.
The submit name intent extracts the person name entity.
The submit return date entity extracts a date-time entity.
The advantage of predefined entities is the saved time and the 
advanced features.
The date-time entity is able to match inputs like Monday, or tomorrow to 
a valid an correct date by default.
The year is also automatically added when the date e.g. first Januarry lies in the past.
This is important for the vacation use case when the vacation is requested for 
the next year. 


\makefigure{luis_interface}{LUIS Web Interface}



\section{Azure Deployment}
\section{Frontend}
\section{Webhook}
\section{Local Deployment (maybe)}