\section{Dialogflow}
Dialogflow\cite{dialogflow} formerly known as API.ai is the chatbot framework of Google and 
is hosted in the Google cloud.
To create a new agent the main language needs to be selected and the name of the agent needs to be defined.
The selected language is the interaction language of the users.
Dialogflow offers the option to create a mega agent by combining other agents.
The web interface of Dialogflow is shown in Figure \ref{fig:dialogflow_interface} where 
the menu shows the intents and entities tabs.
Dialogflow supports the basic concepts of intents, entities, and utterances.
Hence, Table \ref{tab:conversation_data} is used for the implementation 
of the intents, entities, and utterances.
\subsection{Intents and Entities}
When a new intent is created some options are present.
A Dialogflow intent has training phrases, actions, parameters,
a response list, a context, and the fulfillment.
The training phrases are a list of utterances a user would enter.
Parameters are the slots that need to be filled in the conversation.
The parameters for the sickness intent are the rows of Table \ref{tab:example_sentences} where the entity cell is filled with text.
The response list is a list of utterances which is used to respond to the users question.
The context is used to pass parameter values to other intents.
In the fulfillment section, the webhook can be enabled.
By default, an intent is handled by Dialogflow without communication to another service.
In the fulfillment section, a webhook call can be enabled.
For the sickness intent, two parameters are required.
The name of the person is a non-optional parameter,
the return date is an optional parameter.
Dialogflow provides many predefined entities and intents.
The date and person entities are predefined by Dialogflow and it's 
not necessary to create them.
Additionally, yes and no intents are also available and can be used and 
adjusted for the current sickness notification use case.
The final structure is a sickness intent which is followed by either 
the yes or the no intent.
For this, the follow-up intent concept is used where follow-ups can 
be entered.
At the end of the no intent the chatbot responds with a retry response.
At the end of the yes intent, the bot responds with an all done response.
This ends the conversation of the sickness use case and the yes and no intent are marked as the conversation ends.
The process looks the same for the vacation use case.
The vacation intent needs to be created and has a follow-up intent 
yes and no to confirm the user input.
The entities required are a person and a date-span.
Dialogflow offers a predefined entity called date-period which offers this 
functionality and it's not required to create one.
The date and date-period entities offer advanced features.
A valid date period is for instance Monday to Friday.
This is translated to the correct values if e.g. Monday is in the past
the resulting date period will be from the next Monday to Friday 
which follows next Monday.
For the example sentences, the entities can be used to create alternative ways to fill the slots.
The user can be prompted slot after slot or enter them in the first request to speed up the process for more experienced users.
In a sentence like "Max Power is sick and will return on Monday" all the required information is 
already in the sentence. 
The entities need to be defined in the training sentences.
Then the system learns that there are multiple ways to acquire the information.
With suitable training sentences, the chatbot can extract the information directly and jumps directly to the confirmation.  
In general, custom entities are defined as a list of entries or a list of entries with 
synonyms.
An advanced feature of Dialogflow is small talk which can be enabled in the settings.
When small talk is enabled the bot is able to respond to requests like "What are you",
"Bye", and "sorry".
This is another useful feature which is predefined and requires no development effort
and can be enabled with one click. 

\subsection{Webhook}
The Dialogflow webhook requires a HTTP POST URL as an entry point.
The communication relies on JSON as a format.
The entry point of the webhook is shown in Listing \ref{lst:dialogflow_entry_point}.
A request like in Listing \ref{lst:webhookExample} can be accessed like a dictionary or 
an array shown in Listing \ref{lst:dialogflow_request_params} where the name of the 
intent gets extracted. 
After the intent has been extracted the parameters are extracted to process the request.
In the error case, a new fulfillment message is returned.
A fulfillment message as return value prevents follow up intents.
In the success case, an empty response is returned and the bot continues normally.
In Listing \ref{lst:dialogflow_intent_handling} a skeleton for intent handling
is shown for a success and error scenario for an intent called "Sickness Confirmed".

\begin{lstlisting}[caption={Dialogflow Webhook Entry Point}, label={lst:dialogflow_entry_point},captionpos=b,frame=single,language={Python},commentstyle=\color{mygreen},keywordstyle=\color{blue},
    morekeywords={}]                
@restService.route('/dialogflow/webhook', methods=['POST'])
def dialogflowRequestEntryPoint():
    return asJsonResponse(dialogflowRequestHandler()), 200
\end{lstlisting}  

\begin{lstlisting}[caption={Dialogflow Request Parameters}, label={lst:dialogflow_request_params},captionpos=b,frame=single,language={Python},commentstyle=\color{mygreen},keywordstyle=\color{blue},
    morekeywords={}]                
req = request.get_json(force=True)
intent = req.get('queryResult').get('intent').get('displayName')
\end{lstlisting}  

\begin{lstlisting}[caption={Dialogflow Intent Handling}, label={lst:dialogflow_intent_handling},captionpos=b,frame=single,language={Python},commentstyle=\color{mygreen},keywordstyle=\color{blue},
    morekeywords={}]                
if intent == 'Sickness Confirmed':
 params = req.get('queryResult').get('outputContexts')[0].get('parameters')
 employee_name = params.get('employee').get('name').lower()
 return_date = params.get('return_date')
 if !employeeExists(employee_name):
  return {
    "fulfillmentMessages": [
      {"text": {"text": ["Employee not found. Enter sick to retry."]}}
    ]
  } # RETURN ERROR MESSAGE
  else:
   db.store_sickness(employee_name, return_date) # PROCESS DATA
   return {} # CONTINUE NORMALLY
elif intent == 'Vacation':
    ...
\end{lstlisting}  
\makefigure{dialogflow_interface}{Dialogflow Web Interface}

% ----------------------------------------------------
% ---------------------- WATSON ----------------------
% ----------------------------------------------------
\section{Watson Assistant}
Watson Assistant\citet{watsonassistant} is the cloud chatbot technology of IBM
hosted in the IBM cloud.
An IBM cloud account is required to use the services.
First, a new assistant needs to be created which requires a name.
Then dialog skills can be created.
For the sickness notification process, a sick dialog is created.
The web interface of Watson Assistant is shown in Figure \ref{fig:watson_interface}.
The interface offers the concepts intent, entity, and dialog.

\subsection{Intents and Entities}
The required entities are person and date.
Watson Assistant offers system entities for persons and dates.
In the settings under options, a webhook URL can be entered.
The person entity is marked deprecated and is available in English but not in German.
This means a person entity needs to be defined as a replacement for the person entity using entity annotations like recommended by IBM.
To do this every occurrence of person names in the test sentences need to be highlighted 
and marked as person.
For the sickness use case, a new intent needs to be created.
Then the training phrases need to be added.
It is possible to create new entities by using examples or regular expressions.
After the intent had been defined a new dialog needs to be created.
An intent needs to be selected for a dialog.
The dialog is triggered when the intent condition is met.
In this case, the intent is the sickness intent.
In the settings menu, the dialog can be customized.
A customization is necessary to enable slot filling and 
the webhook call which are both necessary for the sickness 
use case.
The required slots are person of type person and the return date of type date.
The person is mandatory and the date is optional.
After the information has been collected the webhook is called.

\subsection{Dialog and Webhook}
The format of the webhook call is JSON.
The parameters need to be specified by hand.
The required parameters are the name of the intent, the person name, and the 
return date.
The name of the intent is necessary to execute the correct action at the webhook.
The name and return date are the information that needs to be stored.
It can be seen that no meta-information is sent in the request of Listing \ref{lst:watson_request}.
The information and format submitted to the webhook is defined by the developer.
The returned information from the webhook is stored as a context variable in Watson Assistant and can be accessed like a struct in C++. 
The format for the webhooks response is also up to the developer.
Based on the webhook response the dialog can go different ways.
If the person was found and a date was entered the date needs to be displayed in the chatbot response.
If the person was found but no date was entered the chatbot response needs to ignore the date.
Whenever a person was found by the webhook the user needs to confirm that the entered information is correct.
Whenever the person wasn't found the user needs to be prompted to retry.
These cases need to be implemented with the conditional statements of Watson Assistant.
Table \ref{tab:watson_cond_response} shows the conditions, responses, and actions that need to be implemented for the use case.
Chatbot responses can be picked sequentially, at random, or as a multiline response.
The multiline response displays all response lines at once.
The sequential version picks them in order and the random variant picks the response at random from 
the available responses.
The confirmation logic needs to be implemented next.
No system entity or dialog for the confirmation logic is present.
This can be implemented with two intents or an entity.
An entity with two possible values is a logical approach.
One value is true the other is false.
The entity is called confirmation.
For these two cases, synonyms are defined like confirm, yes, and OK for true.
When the entity is detected the correct path can be selected with a condition.
Similar conditional logic was used for the webhook response shown 
in Table \ref{tab:watson_cond_response}.
A new dialog node needs to be created for the confirmation or abort 
logic.
There is a jump option in the settings of the dialog node.
In the success cases of the sickness dialog a jump to the 
newly created confirm or abort dialog needs to be entered.
A slot needs to be filled and a webhook call is necessary to store the 
collected information.
These two settings also need to be enabled for the confirm or abort dialog.
The slot is of type confirmation entity and is required.
After the information has been gathered a request is sent to the webhook.
The two possible scenarios are that the information is stored or 
something goes wrong.
If the information has been stored successfully an all done message is displayed.
Else a generic message is displayed indicating that something went wrong.
This ends the sickness dialog.
For the vacation use-case, a date span is needed.
No predefined entity exists for date-spans but a date entity is present.
It's not possible to create entities containing entities with Watson.
The options are to create an intent with two dates in it or a dialog node 
which fetches the required data.
The dialog node approach is chosen because it comes close to the 
behavior implemented with Dialogflow.
A new dialog node named vacation is created.
For the dialog node, an intent called vacation needs to be created with 
the example sentences.
The dialog needs a mandatory person slot and two mandatory date slots.
The start and return dates of the vacation.
After the information has been collected the information can optionally 
be validated through the webhook.
Then the user needs to confirm the information like in the sickness use-case.
Then the final request is sent to the webhook which stores the 
collected information.
The final all done message is displayed and the vacation use-case is finished.

\begin{table}[H]
    \centering
    \begin{tabular}{ l | l | l  }
  Condition & Chatbot Response & Jump To Action \\ \hline \hline
  \$response.OK \&\& \$date!=null & Please confirm \$person is sick & \multirow{2}{*}{Confirm or abort} \\
        & until \$date. & \\ \hline
  \$response.OK & Please confirm \$person is sick. & Confirm or abort \\ \hline
  \multirow{2}{*}{\$response.ERROR} & \$person not found. Enter sick to & \multirow{2}{*}{Wait for input} \\ 
  &  retry. & \\ \hline

\end{tabular}
    \caption{Conditional Responses of Watson Assistant} \label{tab:watson_cond_response}
\end{table} \noindent


\begin{lstlisting}[caption={Watson Assistant Request Format}, label={lst:watson_request},captionpos=b,frame=single,language={Python},commentstyle=\color{mygreen},keywordstyle=\color{blue},
    morekeywords={}]                
[{
        "intent": "Sick",
        "person_name": "Anna Nass",
        "return_date": "2020-05-27"
}]
\end{lstlisting}  

\makefigure{watson_interface}{Watson Assistant Web Interface}

% ----------------------------------------------------
% ----------------------  RASA  ----------------------
% ----------------------------------------------------
\section{Rasa}
Rasa is an open-source framework for NLU and chatbots.
Rasa\cite{rasa} can either be used as a Docker container or 
it can be installed locally.
Rasa uses Python as a development language and can be installed using pip.
A Docker compose file for Rasa and the action server is shown in Listing 
\ref{lst:rasa_compose}.
Rasa is split into a natural language and logic part.
The natural language part takes care of everything which is related to 
the conversation like intents, entities, the story, and the conversation flow.
The second part handles the slot filling and the actions.
For each part, a Docker image can be created.
The Docker compose file of \ref{lst:rasa_compose} can be used in similar fashion 
for the Azure deployment in combination with the images.
Azure also offers a Docker registry to upload and pull custom images.
It's important to choose a compatible version of rasa and rasa-sdk.
The Rasa image is responsible for the NLU part and the Rasa sdk runs the 
actions server.
\begin{lstlisting}[caption={Rasa Docker Compose File}, label={lst:rasa_compose},captionpos=b,frame=single,language={Python},commentstyle=\color{mygreen},keywordstyle=\color{blue},
    morekeywords={}]                
version: '3.0'
services:
    rasa:
    image: rasa/rasa:1.4.6
    ports:
        - 80:5005
    networks: 
        - app_net
    volumes:
        - ./:/app
    command:
        run 
        --enable-api
        --cors *
    
    action_server:
    image: rasa/rasa-sdk:1.4.0    
    volumes:
        - ./actions:/app/actions
    expose: 
        - 5055
    networks: 
        - app_net
networks: 
    app_net:
\end{lstlisting}  
The information for a Rasa chatbot is split in multiple files.
The nlu.md file stores the intents with the utterances used for training.
The intent format is shown in Listing \ref{lst:rasa_intent_format}.
\begin{lstlisting}[caption={Rasa Intent Format}, label={lst:rasa_intent_format},captionpos=b,frame=single,language={Python},commentstyle=\color{mygreen},keywordstyle=\color{blue},
    morekeywords={}]                
## intent:sick
    - My colleague [Max Power](PERSON) is sick until [Friday](date)
    - somebody is sick 
    - a colleague is ill today
\end{lstlisting} 
The stories.md file stores the stories. 
A story defines the flow of the dialog.
The sickness story is shown in Listing \ref{lst:rasa_story_format}.
\begin{lstlisting}[caption={Rasa Story Format}, label={lst:rasa_story_format},captionpos=b,frame=single,language={Python},commentstyle=\color{mygreen},keywordstyle=\color{blue},
    morekeywords={sickness, submit, affirm}]                
## Sickness Affirm
* sickness
    - sickness_form
    - form{"name":"sickness_form"}
* submit{"PERSON":"Anna Maria Mayer"}
    - sickness_form
    - slot{"PERSON":"Anna Maria Mayer"}
* submit{"time":"2020-05-30T00:00:00.000-07:00","DATE":"today"}
    - sickness_form
    - form{"name":null}
    - slot{"time":"2020-05-30T00:00:00.000-07:00"}
* affirm
    - action_submit_sickness_data
    - action_restart
\end{lstlisting} 
All the actions are defined in actions.py.
The sickness form slot filling is an action for instance.
The intents, entities, and actions need to be listed in the domain.yml file.
Otherwise, they can't be used. 
The defined actions can be simple utterances or something complex like the slot filling of a form.
In utterance actions, the response is chosen at random.
In the endpoints.yml file, the action server needs to be entered to use the actions server for slot filling.
The actions server runs separately from the NLU service at the standard port 5055.
The NLU service is accessible at default port 5005.
Rasa offers great freedom when it comes to configuration.
This is important because Rasa does not offer predefined entities as such.
But spacy does offer the required date and person entities for the use cases.
In the config file shown in Listing \ref{lst:rasa_config} the pipeline includes the spacy entity extractor to enable the use of predefined entities.
\citet{spacyapi} lists the supported entities under named entity recognition.
The entity recognition of spacy does work but the dates are stored as they are.
The best case is when the date is converted in a standard format like 
Dialogflow and Watson Assistant do.
To use a more powerful date parsing the Duckling\cite{duckling} Rasa Docker image is used.
Duckling converts entered dates and times to a standard format as the cloud services do.
To add duckling to the project it needs to be added under pipeline in the config file.
It recognizes strings like "next Monday" as dates and converts them correctly.
The person entity of Spacy works as expected.
The used entities need to be added under entities in the domain.yml file.
People with machine learning experience can also define their own pipelines or adjust parameters.
The detailed pipeline is listed in the appendix in Listing \ref{lst:spacy_pipeline_detail}.
The config file defines policies.
The form policy enables the use of forms and the fallback policy enables a fallback action which is triggered when the confidence score is below the defined threshold. 
In the sickness use case, a generic utterance is the fallback response. 


\begin{lstlisting}[caption={Rasa Configuration}, label={lst:rasa_config},captionpos=b,frame=single,language={Python},commentstyle=\color{mygreen},keywordstyle=\color{blue},
    morekeywords={language, pipeline, policies, name}]                
language: en
pipeline: 
    - name: "SpacyNLP"
    - name: "SpacyTokenizer"
    - name: "SpacyFeaturizer"
    - name: "RegexFeaturizer"
    - name: "DucklingHTTPExtractor"
        url: "http://localhost:8000"
    - name: "SpacyEntityExtractor"
    - name: "EntitySynonymMapper"
    - name: "SklearnIntentClassifier"
policies:
    - name: MemoizationPolicy
    - name: TEDPolicy
    - name: MappingPolicy
    - name: FormPolicy
    - name: FallbackPolicy
    nlu_threshold: 0.4
    core_threshold: 0.3
    fallback_action_name: action_default_ask_rephrase
\end{lstlisting} 
The general command required to work with Rasa are train, shell, run actions, and x.
To train Rasa the command "rasa train" is used. 
The shell command starts an interactive shell where the bot can be tested.
The action server needs to be run separately.
If Rasa-X is installed the command "rasa x" starts a web UI shown in Figure \ref{fig:rasa_x_interface}.
The web interface offers the discussed files in the training section with additional
features like the conversation graph (right) called flow for a story (middle) shown in Figure \ref{fig:rasa_x_interface}.
Rasa X offers interactive training where the chatbot is built through live interaction 
with the bot. 
Rasa stores a model every time it's trained.
These models can be exchanged in the models section.
The whole history is stored which can be used for features.
If the feature does not work out the model can be set back to 
the original version.
\makefigure{rasa_x_interface}{Rasa X Interface}
For the sickness use-case, a new intent called sickness needs to be defined in the 
domain file.
The sickness intent needs slot filling and the required entities are the 
person and the return date. 
The slots need to be defined in the slots section.
Every slot needs an utterance which is named utter\_ask\_<SLOTNAME>.
The utterance is displayed when the bot fills a slot.
For the slot filling an action needs to be defined.
An example slot fill action is shown in Listing \ref{lst:slot_fill_action}.
The action needs the name function which returns a unique name.
The required slots need to be defined.
If a slot is required it needs to be filled.
The slot mappings function defines the type of the slots.
The submit function is called as soon as the slots have been filled.
In the submit function the collected information can be sent to the 
external endpoint for instance.
Custom validation rules can be defined for each slot.
This can be used to validate that the entered person name exists.
Now the intent needs to be defined in the NLU file.
The sickness intent and the training phrases need to be defined in the 
NLU  file in the format of \ref{lst:rasa_intent_format}.
The entities are marked with square brackets and the types with parentheses.
Now the sickness story needs to be defined.
When a phrase is recognized which is classified as sickness intent then 
the slot filling needs to be triggered.
The next step is the confirmation dialog where the user confirms the entered information.
A submit sickness data action is required.
To classify the response the intent affirm and deny are created with some example phrases.
After affirm is recognized the confirmation action needs to be triggered.
In the run method of the submit action, the data can be validated and processed.
If deny is recognized the slots need to be cleared and a retry message needs to be displayed.
The sickness story for the successful cases is shown in Listing\ref{lst:rasa_story_format}.
The vacation use case is not different from the sickness use-case in Rasa.
The Duckling time entity can process times, time-spans, dates, and date-spans.
For Duckling there is no difference between a single date and a date-span.
Both are stored in the time slot and are of type time.
Hence, the same person and time slot of the sickness use-case can be used for the 
vacation use-case.
The response messages need to be adjusted.
For the vacation use-case, a new intent needs to be created in the domain file.
The vacation training phrases need to be listed in the NLU file.
A form and a submit action need to be created for the vacation intent.
The structure can be copied from the sickness use-case.
Then the response messages and conditions are adjusted as needed.
The actions need to be listed in the domain file to be usable in the stories.
Then the new stories need to be created.
With Rasa-X the stories are created with live interaction with the bot.
The stories are structured like in the sickness use-case.








% ----------------------------------------------------
% ----------------------  LUIS  ----------------------
% ----------------------------------------------------
\section{LUIS}
With LUIS\cite{luisdocs}, the language understanding service of Microsoft,
NLU services can be created.
NLU is only a part of a chatbot and it's not possible 
to create a chatbot with LUIS alone.
The chatbot features are available through the Microsoft bot services 
in combination with LUIS.
A LUIS account is necessary to create NLU services.
When a new project is created the name and a language need to be defined for 
the NLU application.
The chosen language is the language the users will use for the communication with the 
LUIS application.
The modification is done in the build tab.
The interface shown in Figure \ref{fig:luis_interface} offers intents and entities
like described in Chapter \ref{chap:basics}.
To create the NLU part with LUIS Table \ref{tab:conversation_data} will be used.
\ref{tab:conversation_data} defines the intents entities and response utterances
that need to be created in LUIS.
Figure \ref{fig:conversationflow} has no influence on LUIS because the 
conversation handling is not part of the NLU service.
To create an intent only a name and a list of training phrases (utterances) are necessary.
Multiple options are available to create different kinds of entities.
There are simple entities like e.g. a city.
They describe exactly one concept.
There are composite entities that are built from multiple entities (parts).
E.g. a ticket order entity could consist of an order number, the number of 
tickets, and the event.
An entity list is used when the valid items are limited like in the book hotel example.
There is only a limited number of valid hotel names and the list holds all valid 
hotels.
The invalid hotels are uninteresting in such a case.
Regular expressions\cite{regex} can also be used to define an entity.
It is also possible to add prebuilt entities and prebuilt domain entities.
The entities required for the sickness intent are person name and date. 
LUIS offers the prebuilt entities date-time and person-name.
This means no new entities need to be created for the sickness use-case
since the are prebuilt.
To test the NLU functionality two additional intents need to be created to 
simulate the conversation.
A submit name and a submit return date intent are required.
The submit name intent extracts the person name entity.
The submit return date entity extracts a date-time entity.
The advantage of predefined entities is the saved time and the 
advanced features.
The date-time entity is able to match inputs like Monday or tomorrow to 
a valid and correct date by default.
The year is also automatically added when the date e.g. first January lies in the past.
This is important for the vacation use case when the vacation is requested for 
the next year. 


\makefigure{luis_interface}{LUIS Web Interface}




\section{Frontend}
The main purpose of the frontend is message handling.
It tracks the user input, submits the input to the chatbot service 
when the submit button is pressed, 
and displays the response.
A classic message in messengers has the properties sender, timestamp, and message.
The message format of the frontend is shown in Listing \ref{lst:message_format}
\begin{lstlisting}[caption={Message Format}, label={lst:message_format},captionpos=b,frame=single,language={Python},commentstyle=\color{mygreen},keywordstyle=\color{blue},
    morekeywords={}]                
messages.push({
    sender: "<Sender>",
    message: "<Message>",
    timestamp: this.getTimestamp()
})
\end{lstlisting}  
A toolbar with title, message area, and submit area are necessary for a messenger UI.
The structure is shown in Listing \ref{lst:messenger_structure}
\begin{lstlisting}[caption={Messenger Structure}, label={lst:messenger_structure},captionpos=b,frame=single,language={Python},commentstyle=\color{mygreen},keywordstyle=\color{blue},
    morekeywords={MessengerToolbar, MessageArea, Divider, MessengerSubmitArea, Box}]                
<Box width={"25em"}>
    <MessengerToolbar title="3 Banken IT Dev Chatbot" />
    <MessageArea entries={this.state.messages} />
    <Divider light />
    <MessengerSubmitArea 
        message={this.state.message} 
        textHandler={this.handleMessage.bind(this)} 
        submitMessage={this.handleSubmit.bind(this)} 
    />
</Box>
\end{lstlisting}  


The UI of the React messenger frontend is shown in action in Figure \ref{fig:messengerui}.
\section{Webhook}
