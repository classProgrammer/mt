
\section{Framework Comparison}

\subsection*{Portability}
An important factor for software systems is the portability.
Portability is defined in ISO 25010 \cite{iso25010}.
As a measurement criterion, the number of providers that support the technology is defined.
A subfactor is how tied to a provider a technology is.
When it comes to the chatbot cloud services (LUIS, Dialogflow, IBM Watson Assistant)
the provider can't be changed.
All three services run in the cloud of the provider and nowhere else.
The data processing is done in the cloud and can't be done locally or on a company server.
This means that the three cloud services are not portable at all and 
the provider can't be changed.
With Rasa things are different.
It is a local solution which means it can run on a local machine and a company server.
The data processing is also done locally and never goes into the cloud of a provider.
Docker images for Rasa are provided too.
It's possible to run Docker containers in the clouds of all big providers.
This means Rasa can run everywhere where Docker or a Python environment are possible 
and the provider can be changed easily.
Another aspect of portability is the number of devices where the chatbot can be developed
from.
The cloud frameworks are accessible via the internet and can be used 
and developed from every device which has an internet connection and a
browser.
Rasa can only be developed on machines which have access to the 
source files.
It is per default not accessible via the internet.
However, it can be deployed in a cloud environment.
Then it can be used like the cloud services but it 
still can't be developed easily on any device.
The third aspect of portability is the migration from one technology to another.
Rasa offers an import functionality (beta version) which allows The
import of Dialogflow, LUIS, Wit.ai, and IBM Watson chatbots.
In general, all chatbot frameworks use JSON format to store
the training data.
But evert technology saves information differently and it's not 
easy to migrate an existing chatbot to another technology.

% was it possible to implement the design with the technologies?
\subsection*{Implementation of the Design}
like \citet{singhbuilding} said, the best evaluation criteria is if a task can be achieved with a technology or not.
The goal is to implement the dialogs shown in Figure \ref{fig:conversationflow}.
The result of this evaluation is a binary criterion which is either true when the implementation was possible or false
if it wasn't possible.
The prototype implementations in Chapter \ref{chap:proto} shown that the dialogs can be 
implemented with Dialogflow, Watson Assistant, and Rasa.
It is not possible to implement the dialogs with LUIS. 
This was to be expected since LUIS is a NLU service and not a chatbot technology and offers no dialog control.
The Microsoft Bot framework is the chatbot technology which uses LUIS but 
is not considered in this thesis.
The implementation of the dialogs of Figure \ref{fig:conversationflow} is possible with all 
three chatbot technologies.


% project complexity
\subsection*{Project Setup Complexity} \label{sec:setup_complex}
% required tools
As a measurement criterion, the required tools will be used.
To set up Watson Assistant, Dialogflow, and LUIS a web-browser is needed.
Those three technologies require nothing else since the development environment,
files, and data are built into the websites of the technologies.
This means only one tool is required.
For Rasa three approaches are evaluated.
Rasa can be installed locally or with Docker and Rasa-X can be used for development.
Rasa-X provides a web-interface similar to the cloud technologies.
A local setup requires a Python environment and a package manager like pip.
A text editor or IDE is required to edit the files.
This results in a count of at least three.
The second way is a Docker container setup.
For this approach, Docker and a text-editor are required and results in a count of two.
With Rasa-x a web-interface replaces the text-editor but the score remains the same.
This means the could technologies have a score of one and Rasa has a score of two or three
depending on the approach.
% setup time
As second measurement criteria for setup complexity a setup time ranking will be used.
To create a new project from scratch with the cloud technologies only an account is required.
Then a new project can be created instantly.
No tools or additional software needs to be installed to have everything development ready.
For Rasa, the tools and the environment needs to be created.
For the development, either Rasa needs to be installed or the Docker image needs to be pulled.
The setup for Rasa takes more time because the development environment needs to be provided.
This results in the first place for cloud services and the last/second place for Rasa in the
setup time comparison. 

% development complexity
\subsection*{Development Complexity}
The easiest framework is LUIS since it is a NLU framework it doesn't
offer dialog handling mechanisms.
The development complexity for the NLU part is compareable to Dialogflow and 
Watson Assistant.
They use the same concepts (intent, entity, utterance) in the UI and the 
mechanisms to create intents and entities are the same.
The easiest chatbot framework is Dialogflow since it offers a simpler UI 
than Watson Assistant, more predefined entities, and the option to 
include features like small talk.
Watson Assitant UI is close to Dialogflow but is not as transparent 
and additional logic can be handeled inside of Watson which is a
useful feature at times but makes the application more complex.
Rasa is the most complex chatbot and NLU framework.
The creation of intents, entities and utterances is scattered across multiple files.
By default form filling actions are not supported and need to be created using Python.
Luis, Dialogflow and Watson Assistant offer form filling by default and it can be enabled with a 
few clicks in the UI without a programming language.
Rasa doesn't offer any predefined entities  by default.
To add predefined entities the pipeline needs to be adjusted which requires additional effort 
from the user.
The creation of stories is also hardest with Rasa since the form filling action for instance 
requires additional information iside of the story, shown in Listing \ref{lst:rasa_sickness_story}
where the the form, person and time slots need to be set correctly.
Trying to set the forms and slots by hand is not recommended.
The stories should be created using live training.
It is a useful feature to create natural conversations but it's also more complex than the 
creation of stories with the dialog node systems of Watson Assistant and Dialogflow.
The dialog node system of Dialogflow is shown in Figure \ref{fig:dialog_nodes}.
% dev complexity based on the use-case
When the framework development complexity is evaluated based on the use-cases of this thesis 
the predefined entities are very important.
The creation of new entities is equally hard with each framework.
Hence, the more of the required entities are predefined the easier the development process gets.
Dialogflow offers all three required entities for the German and English language as does 
Rasa.
Watson Assistant doesn't offer a person enity in German and the one for the English language is marked depricated.
It also has no date-span entity but it can be replaced by two seperate dates.
This means that Watson Assistant offers two of the three entities for both required languages. 
LUIS offers all three entities for the English language but has no person entity for the German language. 



% price
\subsection*{Pricing} \label{sec:pricing}
Dialogflow can be used by companies for free.
The limits for the free tier are 180 text requests per minute.
With the Essential and Plus version, 600 requests per minute are possible.
Rasa is a free to use open-source tool and costs nothing.
There is an enterprise edition with some benefits but the price
is calculated individually.
Watson Assistant offers a free tier with up to 1.000 users per month, 
10,000 messages per month.
The free tier of the Microsoft Bot Framework offers five requests per second and 
10,000 messages per month which is the same amount of messages IBM offers.
Five requests per second are 300 requests per minute which is more than the 
free tier of Dialogflow offers.
A user is a person who interacts with the bot at least once.
The Plus edition costs \$120 for 1.000 users per month with
no message or user limit.
The pricings and editions of the framework are shown in Table \ref{tab:pricing}.
The price of Dialogflow is based on the number of requests,
IBM offers a fixed number of requests and users, and with Rasa the developer has
to take care of the number of users and requests in soft- and hardware.
The Microsoft Bot Framework charges per 1,000 users like IBM does.
The price of the Microsoft Bot Framework is calculated by combining the prices 
for the framework and LUIS.
The resulting cost for the Microsoft Bot Framework and LUIS is \$0.002 per request 
which is the same Dialogflow charges for the enterprise essential edition.  
The standard editon also offers 50 transactions per second which 
are 3,000 requests per minute which is far more than the 600 requests per second of 
Dialogflow.
In the free tier, Dialogflow provides the best value because there is no message or user limit.
In the priced tier Watson costs \$0.12 per user per month (\$120 for 1.000 users per month).
Dialogflow and the Microsoft Bot Framework using LUIS charge \$0.002 per request each.
Watsons cost of \$0.12 per user per month equals 60 requests per user per month with Dialogflow or 
the Microsoft Bot Framework plus LUIS.


\begin{table}[H]
    \centering
    \begin{tabular}{ l | l | l }
        Framework & Edition & Price \\ \hline \hline
        \multirow{3}{*}{Dialogflow} & Standard & Free \\
        & EE Essentials &  \$0.002 per request\\
        & EE Plus & \$0.004 per request \\ \hline

        \multirow{3}{*}{Watson Assistant} & Lite & Free \\
        & Plus &  \$120 for 1,000 users/month\\
        & Premium & Individual \\ \hline

        \multirow{3}{*}{Rasa} & Rasa Open Source & Free \\
        & Rasa X &  Free\\
        & Rasa Enterprise & Individual \\ \hline

        
        \multirow{3}{*}{Microsoft Bot Framework} & Free  & Free \\
        & \multirow{2}{*}{S1} & \$0.50 per 1,000 requests \\ 
        & & \$0.0005 per requests \\ \hline
                
        \multirow{3}{*}{LUIS} & Free  & Free (10.000 requests per month) \\
        & \multirow{2}{*}{Standard} &  \$1.50 per 1,000 transactions\\
        & &  \$0.0015 per request
    \end{tabular}
    \caption{Pricing of Frameworks \cite{rasa, dialogflow, watsonassistant,luisdocs}} \label{tab:pricing}
\end{table} \noindent


% needed knowledge
\subsection*{Learnability}
The easiest of the chatbot frameworks is Dialogflow.
Compared to Rasa the setup is easy.
The UI is minimalistic and easy to understand.
It's easier to develop a chatbot with Dialogflow than with Watson Assistant or Rasa.
The Watson Assistants UI is intransparent when compared to Dialogflow and a developer 
may need more time to orient.
Rasa requires the most time to develop a chatbot and to get everything ready.
As described in Section \ref{sec:setup_complex} the setup of cloud services is 
easy since the cloud providers take care of the setup.
Rasa has to be installed or used as a Docker container which is harder than the cloud 
setup and requires more time.
To use Watson and Dialogflow no programming skills are required while they are needed for 
Rasa which makes it harder to learn.
It's also harder to define a dialog structure with Rasa since a story 
needs to be written in the format shown in Listing \ref{lst:rasa_story_format}
whereas with Dialogflow and Watson dialog nodes are created which 
can be strucutred in a GUI with drag-and-drop.
Furthermore, it's harder to create form filling actions in Rasa.
In Rasa the form filling needs to be defined through code like in Listing \ref{lst:slot_fill_action} while 
Dialogflow and Watson support it through a simple GUI shown in Figure \ref{fig:formfill}.
When the GUI of Rasa-X is used defining new entities and intents works the same way for the three frameworks.
To enable the use of predefined entities the pipeline of Rasa needs to be modified which is more 
complicated than with Dialogflow where they are enabled by default and Watson where the entities need to be 
enabled in the settings.
% predefined types/entities
Another measurement criteria for learnability is the number of predefined entities.
If a predefined entity is available for a given problem a developer doesn't need to 
invest time in creating an entity or collecting training data. 
Table \ref{tab:predefined_entities} shows the amount of predefined entities 
present for the tested technologies.
Dialogflow provides the most predefined entities, followed by LUIS.
Compared to the two above the other technologies provide a small amount of predefined entities.
\begin{table}[H]
    \centering
    \begin{tabular}{ c | l | c }
        Rank & Framework &  Entities \\ \hline \hline
        1 & \multirow{1}{*}{Dialogflow} & over 400 \\
        2 & \multirow{1}{*}{LUIS} & circa 150 \\
        3 & \multirow{1}{*}{Spacy} & 18 \\
        4 & \multirow{1}{*}{Duckling} & 11 \\
        5 & \multirow{1}{*}{Watson} & 7  \\
        6 & \multirow{1}{*}{Rasa} & 0 \\
    \end{tabular}
    \caption{Predefined Entities} \label{tab:predefined_entities}
\end{table} \noindent
% deployment
\subsection*{Deployment Complexity}
A major factor for the development of a chatbot is where it can be deployed and how easy the 
deployment process is.
For the cloud technologies the deployment is done by the provider and there is no development overhead for 
the deployment.
The deployment of Rasa needs to be done by the developer.
Rasa can either run in a Docker container or it can be installed in a Python environment.
The Docker container approach was choosen for the deployment because it's supported by 
all cloud providers.
The cloud technologies share the first place in the deployment ranking since there is no overhead for the 
developer.
The only technology tested which produces deployment overhead is Rasa. 
The second deployment is the local deployment.
It's possible to deploy Rasa locally using Docker or a Python environment.
As expected it's not possible to deploy a cloud chatbot locally or in a non native cloud environment.
This ties a user to the provider and is not flexible.
The only technology used in this thesis which supports a local deployment or the deployment in 
more than one cloud environment is Rasa. 

% communication via Rest endpoint, between the webhook, metadata
\subsection*{Communication}
All of the chatbot and NLU technologies offer Rest API endpoints for communication.
There is no difference in the type of communication.
Dialogflow and Watson Assistant communicate with a webhook when the extraction 
of information was successful.
Rasa communicates with the action server which serves a similar purpose as the webhooks.
The communication between the frameworks and the external service is done with 
JSON messages.
While Dialogflow and LUIS offer metadata, shown in Listing \ref{lst:dialogflow_request_params}, \ref{lst:luis_intent_response}
and \ref{lst:luis_entity_response}, Rasa and Watson Assistant offer no metadata by default.
The required information needs to be defined by the developer.
This means more logic can be handeled with Rasa and Watson Assistant on the chatbot side since the 
format and processing of the result is up to the developer.
Especially with Rasa the logic can be freely defined since the action server is written in Python and 
all features of programming languages are available for the result processing.
The behavior of Watson Assistant and Rasa can be mimicked by shifting the logic from the chatbot to the webhook.

%  precision entity recognition/extraction
\subsection*{Entity Recognition and Extraction}

Entity recognition and extraction is a key feature of chatbots.
For the use-cases of this thesis a date, date-span, and person entity are required.
The entity recognition and extraction tests focus solely on the required entities.
In Table \ref{tab:entity_extraction_recognition} the supported entities are listed alongside the technologies.
Rasa does not offer any entities but the pipeline can be adjusted.
One part of the pipeline is Spacy which offers person and date entities.
Spacy does not convert the date into a standard format.
For this reason, Duckling was added to the pipeline.
Duckling focuses on numeric values, dates, and times and converts dates into a standardized format.
Dialogflow offers the system entities date, date-period, and person.
LUIS offers a person and date-time entity.
The date-time entity can also be a date-span.
Hence, no entities need to be created in LUIS or Dialogflow for the use-cases of this 
thesis.
\begin{table}[H]
    \centering
    \begin{tabular}{ c | c | c | c | c | c }
        Framework & Subtype & Person & Date & Date Span & Standardized Output Date \\ \hline \hline
        \multirow{3}{*}{Rasa} & - & \xmark & \xmark & \xmark & \xmark \\
        & Spacy & \cmark & \cmark & \cmark & \xmark \\ 
        & Duckling & \xmark & \cmark & \cmark & \cmark \\ \hline
        Dialogflow & - & \cmark & \cmark & \cmark & \cmark \\ \hline
        Watson & - & \xmark & \cmark & \xmark & \cmark \\ \hline
        \multirow{2}{*}{LUIS}  & English & \cmark & \cmark & \cmark & \cmark \\
         & German & \xmark & \cmark & \cmark & \cmark \\
    \end{tabular}
    \caption{Framework Entity Recognition and Extraction} \label{tab:entity_extraction_recognition}
\end{table} \noindent
Watson Assistant offers the system entity sys-date but no date-span entity.
When Watson recognizes two sys-dates the slots are filled correctly as start and return date
and the result is comparable to a date-span entity.
Dialogflow extracts the date, time, and timezone in the format "2020-06-02T12:00:00+02:00".
Watson and LUIS extract only the date in the format "2020-06-02".
Duckling extracts dates (time entity) in the format "2020-06-07T00:00:00.000-07:00".
The second date of date spans is always set to one day after the recognized day in Duckling.
This means that the second dates of Duckling differ from the dates of the other technologies
shown in Table \ref{tab:date_span_entity_extraction_recognition} in the "Extracted Value" column.
The column "Extracted Value" of Tables \ref{tab:date_entity_extraction_recognition},
\ref{tab:date_entity_extraction_recognition2},
and \ref{tab:date_span_entity_extraction_recognition},
\ref{tab:date_span_entity_extraction_recognition2} show that Spacy doesn't 
convert the dates into standard format.
The date and date-span tests focus on different date inputs.
It is to be expected that a recognized date format works for 
all instances of the format.
If Monday is recognized correctly the other weekdays will work too. 
% person entity
The person entity has been added to Watson Assistant with custom training data.
The expectation is that the person entity of Watson Assistant performs worst because 
it's no system entity and is built with a low amount of training data.
The results of the date recognition and extraction are listed in Table \ref{tab:date_entity_extraction_recognition}
and \ref{tab:date_entity_extraction_recognition2}.
The expectation of the results is listed as true positive(TP) when the date should be recognized and extracted or as true negative(TN)
when the entity shouldn't be recognized and extracted.
The test result shows that the date entity of Dialogflow worked best with just one error
followed closely by Duckling with two errors.
Spacy performed worst with five errors in ten tests followed by Watson Assistant with four errors out of ten.
No technology worked correctly on test case eight of Table \ref{tab:date_entity_extraction_recognition}.
This is true for the english language selection.
When German is selected as a language the performance on dates separated by
"." increases.
It also shows that weekdays and tomorrow work with all technologies.
The extracted value column shows that Spacy recognizes many date inputs correctly but doesn't convert them.

\begin{table}[h]
    \centering
    \begin{tabular}{ c | l | c | c | c | c | c | c | c | c }
        Framework & Type & TP & TN & FP & FN & p & r & F-Score & Tests \\ \hline \hline
        \multirow{4}{*}{\shortstack[l]{Dialogflow\\date-period}} 
        & Date & 8 & 1 & 0 & 1 & 1.0 & 0.888 & 0.94 & 10 \\ 
        & Date-Span & 2 & 0 & 0 & 4 & 1.0 & 0.333 & 0.5 & 6 \\ 
        & Person & 3 & 2 & 0 & 1 & 1.0 & 0.75 & 0.857 & 6 \\ \cline{2-10}
        & Summary & 13 & 3 & 0 & 6 & 1.0 & 0.684 & 0.813 & 22 \\ \hline
        \multirow{4}{*}{\shortstack[l]{Dialogflow\\date-time}}
        & Date & 8 & 1 & 0 & 1 & 1.0 & 0.888 & 0.94 & 10 \\ 
        & Date-Span & 5 & 0 & 0 & 1 & 1.0 & 0.833 & 0.909 & 6 \\ 
        & Person & 3 & 2 & 0 & 1 & 1.0 & 0.75 & 0.857 & 6 \\ \cline{2-10}
        & Summary & 16 & 3 & 0 & 3 & 1.0 & 0.842 & 0.914 & 22 \\ \hline
        \multirow{4}{*}{Watson} 
        & Date & 6 & 0 & 1 & 3 & 0.857 & 0.666 & 0.75 & 10 \\ 
        & Date-Span & 5 & 0 & 0 & 1 & 1.0 & 0.833 & 0.909 & 6 \\ 
        & Person &  2 & 1 & 1 & 2 & 0.666 & 0.5 & 0.571 & 6 \\ \cline{2-10}
        & Summary & 13 & 1 & 2 & 6 & 0.866 & 0.684 & 0.765 & 22 \\ \hline
        \multirow{4}{*}{LUIS} 
        & Date & 6 & 0 & 1 & 3 & 0.857 & 0.666 & 0.75 & 10 \\ 
        & Date-Span & 3 & 0 & 0 & 3 & 1.0 & 0.5 & 0.666 & 6 \\ 
        & Person & 4 & 0 & 2 & 0 & 0.666 & 1.0 & 0.8 & 6 \\ \cline{2-10}
        & Summary & 13  & 0 & 3 & 6 & 0.8125 & 0.684 & 0.743 & 22 \\ \hline
        \multirow{4}{*}{Spacy} 
        & Date & 4 & 1 & 0 & 5 & 1.0 & 0.444 & 0.615 & 10 \\ 
        & Date-Span & 2 & 0 & 0 & 4 & 1.0 & 0.333 & 0.5 & 6 \\ 
        & Person & 4 & 0 & 2 & 0 & 0.666 & 1.0 & 0.8 & 6 \\ \cline{2-10}
        & Summary & 10 & 1 & 2 & 9 & 0.833 & 0.526 & 0.645 & 22 \\ \hline
        \multirow{3}{*}{Duckling} 
        & Date & 7 & 1 & 0 & 2 & 1.0 & 0.777 & 0.875 & 10 \\ 
        & Date-Span & 5 & 0 & 0 & 1 & 1.0 & 0.833 & 0.909 & 6 \\ \cline{2-10}
        & Summary & 12 & 1 & 0 & 3 & 1.0 & 0.8 & 0.888 & 16 \\ \hline
        Dialogflow & Date Sum & 13 & 1 & 0 & 2 & 1.0 & 0.866 & 0.929 & 16 \\ \hline
        Watson & Date Sum & 11 & 0 & 1 & 4 & 0.916 & 0.733 & 0.815 & 16 \\ \hline
        LUIS & Date Sum & 9 & 0 & 1 & 6 & 0.9 & 0.6 & 0.72 & 16 \\ \hline
        Spacy & Date Sum & 6 & 1 & 0 & 9 & 1.0 & 0.4 & 0.571 & 16 \\ \hline
        Spacy + & \multirow{2}{*}{Summary} & \multirow{2}{*}{16} & \multirow{2}{*}{1} & \multirow{2}{*}{2} & \multirow{2}{*}{3} & \multirow{2}{*}{0.888} & \multirow{2}{*}{0.842} & \multirow{2}{*}{0.865} & \multirow{2}{*}{22} \\
        Duckling & & & & & & & & & \\
    \end{tabular}
    \caption{Entity Recognition Result Evaluation} \label{tab:entity_extraction_eval}
\end{table} \noindent
The detailed result of the entity recognition comparison is shown in Table \ref{tab:entity_extraction_eval}.
The column name p stands for precision and r for recall.
Date sum shows the framework result based on the date and date-span recognition
without the person entity for the comparison with Duckling which has no person entity.
The last row of Table \ref{tab:entity_extraction_eval} shows the result for the combination 
of Duckling for dates and date-spans and Spacy for persons.
This makes the comparison of Rasa with the other technologies possible.
The true negatives (TN) are not part of the f-score calculation.
Table \ref{tab:entity_extraction_eval} shows that the technology with the 
best f-score across all tests (0.914) is Dialogflow followed by 
a combination of Duckling for dates and date-spans and Spacy  
for persons is used.
Rasa can use Duckling and Spacy and reaches the second highest f-score.
The best f-score (0.929) for the combination of dates and date-spans was achieved  
by Dialogflow.
Dialogflow has the best date entity (f-score of 0.94) when the date-time entity is used 
instead of the date-period entity.
The date-period entity of Dialogflow reached a low f-score of 0.5 and is not recommended for further use. 
Duckling, Watson, and Dialogflow were on an equal level when only date-spans are considered. 
All three reached an f-score of 0.909.
Dialogflow also has the best f-score for the person entity.
Watsons person entity performed worst as expected.
With more training data the performance of Watsons custom person 
entity can be increased.
In test case two of Table \ref{tab:date_span_entity_extraction_recognition} Spacy 
recognized two separate dates instead of a date-span.
This is counted as a failure because in the application there is only one date slot present
and the information can't be processed correctly.
In test case one Dialogflows date-period entity and Luis convert Monday to a date in the past.
This is now what is needed for the use-cases since it should be the next matching 
weekday and not the last.
Test case two gives a hint that the next occurring weekday is the correct one.
The result is not influenced and Dialogflows date-period entity and LUIS still take the wrong day.
The functionality was retested on another day.
LUIS still failed to extract the correct date but Dialogflow 
managed to do so.
In the test result it will be counted as a failure of Dialogflows date-period entity
since it doesn't deliver reliable results.

\begin{table}[h]
    \centering
    \begin{tabular}{ c | c | c | c | c | c  }
        \multirow{2}{*}{No} & \multirow{2}{*}{Input} & \multirow{2}{*}{Expected} & \multirow{2}{*}{Technology} & Recognized & Extracted \\ 
                 &&          &            & Correctly  & Value     \\ \hline \hline
        \multirow{4}{*}{1} &\multirow{4}{*}{Franz Bauer} & \multirow{4}{*}{TP} 
                                  & Watson & \cmark & Franz Bauer \\
                                  && & Dialogflow & \cmark & Franz Bauer \\
                                  && & Spacy & \cmark & Franz Bauer \\
                                  && & LUIS & \cmark & Franz Bauer \\
                                  \hline
        \multirow{4}{*}{2} &\multirow{4}{*}{franz bauer} & \multirow{4}{*}{TP} 
                                  & Watson & \cmark & franz bauer \\
                                  & && Dialogflow & \cmark & franz bauer \\
                                  & && Spacy & \cmark & franz bauer \\
                                  & && LUIS & \cmark & franz bauer \\
                                  \hline
        \multirow{4}{*}{3} &\multirow{4}{*}{Anna Maria Mayer} & \multirow{4}{*}{TP} 
                                  & Watson & \xmark & Maria Mayer \\
                                  & && Dialogflow & \xmark & Maria Mayer \\
                                  & && Spacy & \cmark & Anna Maria Mayer\\
                                  & && LUIS & \cmark & Anna Maria Mayer\\
                                  \hline
        \multirow{4}{*}{4} &\multirow{4}{*}{Anna Mayer-Bauer} & \multirow{4}{*}{TP} 
                                  & Watson & \xmark & Anna Mayer \\
                                  & && Dialogflow & \cmark & Anna Mayer-Bauer \\
                                  & && Spacy & \cmark & Anna Mayer-Bauer\\
                                  & && LUIS & \cmark & Anna Mayer-Bauer\\
                                  \hline
        \multirow{4}{*}{5} &\multirow{4}{*}{Asdfgh} & \multirow{4}{*}{TN} 
                                  & Watson & \cmark & - \\
                                  & && Dialogflow & \cmark & - \\
                                  & && Spacy & \xmark & Asdfgh\\
                                  & && LUIS & \xmark & Asdfgh\\
                                  \hline
        \multirow{4}{*}{6} &\multirow{4}{*}{Asdfgh Qwert} & \multirow{4}{*}{TN} 
                                  & Watson & \xmark & Asdfgh Qwert \\
                                  & && Dialogflow & \cmark & - \\
                                  & && Spacy & \xmark & Asdfgh Qwert\\
                                  & && LUIS & \xmark & Asdfgh Qwert\\
    \end{tabular}
    \caption{Person Entity Recogniton and Extraction} \label{tab:person_entity_extraction_recognition}
\end{table} \noindent

\begin{table}[h]
    \centering
    \begin{tabular}{ c | c | c | l | c | c  }
        \multirow{2}{*}{No} & \multirow{2}{*}{Input} & \multirow{2}{*}{Expected} & \multirow{2}{*}{Technology} & Recognized & Extracted \\ 
                 &&          &            & Correctly  & Value     \\ \hline \hline
        \multirow{5}{*}{1} & \multirow{5}{*}{Tomorrow} & \multirow{5}{*}{TP} 
                                  & Watson & \cmark & 2020-06-01 \\
                                  && & Dialogflow & \cmark & 2020-06-01 \\
                                  && & Spacy & \cmark & Tomorrow \\
                                  && & Duckling & \cmark & 2020-06-01 \\ 
                                  && & LUIS & \cmark & 2020-06-01 \\ 
                                  \hline
        \multirow{5}{*}{2} &\multirow{5}{*}{Friday} & \multirow{5}{*}{TP} 
                                  & Watson & \cmark & 2020-06-05 \\
                                  & && Dialogflow & \cmark & 2020-06-05 \\
                                  & && Spacy & \cmark & Friday \\
                                  & & &Duckling & \cmark & 2020-06-05 \\ 
                                  && & LUIS & \cmark & 2020-06-05 \\ 
                                  \hline
        \multirow{5}{*}{3} &\multirow{5}{*}{6.June} & \multirow{5}{*}{TP} 
                                  & Watson & \xmark & 2020-06-01 \\
                                  && & Dialogflow & \cmark & 2020-06-06 \\
                                  && & Spacy & \xmark & - \\
                                  & && Duckling & \xmark & - \\ 
                                  && & LUIS & \xmark & 2020-06-01 \\ 
    \end{tabular}
    \caption{Date Entity Recogniton and Extraction} \label{tab:date_entity_extraction_recognition}
\end{table} \noindent

\begin{table}[h]
    \centering
    \begin{tabular}{ c | c | c | l | c | c  }
        \multirow{2}{*}{No} & \multirow{2}{*}{Input} & \multirow{2}{*}{Expected} & \multirow{2}{*}{Technology} & Recognized & Extracted \\ 
                 &&          &            & Correctly  & Value     \\ \hline \hline
        \multirow{5}{*}{4} &\multirow{5}{*}{June 6} & \multirow{5}{*}{TP} 
                 & Watson & \cmark & 2020-06-06 \\
                 && & Dialogflow & \cmark & 2020-06-06 \\
                 && & Spacy & \cmark & June 6 \\
                 & && Duckling & \cmark & 2020-06-06 \\ 
                 && & LUIS & \cmark & 2020-06-06 \\ 
                 \hline
        \multirow{5}{*}{5} &\multirow{5}{*}{6th of June} & \multirow{5}{*}{TP} 
                 & Watson & \cmark & 2020-06-06 \\
                 && & Dialogflow & \cmark & 2020-06-06 \\
                 && & Spacy & \cmark & 6th of June \\
                 & && Duckling & \cmark & 2020-06-06 \\ 
                 && & LUIS & \cmark & 2020-06-06 \\ 
                 \hline
        \multirow{5}{*}{6} &\multirow{5}{*}{06/06} & \multirow{5}{*}{TP} 
                 & Watson & \cmark & 2020-06-06 \\
                 && & Dialogflow & \cmark & 2020-06-06 \\
                 && & Spacy & \xmark & - \\
                 & && Duckling & \cmark & 2020-06-06 \\ 
                 && & LUIS & \cmark & 2020-06-06 \\ 
                 \hline
        \multirow{5}{*}{7} &\multirow{5}{*}{06-06} & \multirow{5}{*}{TP} 
                                  & Watson & \xmark & - \\
                                  && & Dialogflow & \cmark & 2020-06-06 \\
                                  && & Spacy & \xmark & - \\
                                  & && Duckling & \cmark & 2020-06-06 \\ 
                                  && & LUIS & \xmark & - \\ 
                                  \hline
        \multirow{5}{*}{8} &  \multirow{5}{*}{06.06} & \multirow{5}{*}{TP} 
                                  & Watson & \xmark & - \\
                                  && & Dialogflow & \xmark & - \\
                                  & && Spacy & \xmark & - \\
                                  & && Duckling & \xmark & 2020-05-31 \\ 
                                  && & LUIS & \xmark & - \\ 
                                  \hline
        \multirow{5}{*}{9} & \multirow{5}{*}{06.06.2020} & \multirow{5}{*}{TP} 
                                  & Watson & \cmark & 2020-06-06 \\
                                  && & Dialogflow & \cmark & 2020-06-06 \\
                                  & && Spacy & \xmark & - \\
                                  & && Duckling & \cmark & 2020-06-06 \\ 
                                  && & LUIS & \cmark & 2020-06-06 \\ 
                                  \hline

        \multirow{5}{*}{10} & \multirow{5}{*}{06/34/2020} & \multirow{5}{*}{TN} 
                                  & Watson & \xmark & 2020-06-01 \\
                                  && & Dialogflow & \cmark & - \\
                                  && & Spacy & \cmark & - \\
                                  & && Duckling & \cmark & - \\ 
                                  && & LUIS & \xmark & 2020-01-01 \\ 
                                  \hline
    \end{tabular}
    \caption{Date Entity Recogniton and Extraction 2} \label{tab:date_entity_extraction_recognition2}
\end{table} \noindent


\begin{table}[h]
    \centering
    \begin{tabular}{ c | c | c | c | c | c  }
        \multirow{2}{*}{No} & \multirow{2}{*}{Input} & \multirow{2}{*}{Expected} & \multirow{2}{*}{Technology} & Recognized & Extracted \\ 
                 &&          &            & Correctly  & Value     \\ \hline \hline
        \multirow{12}{*}{1} &\multirow{12}{*}{Monday to Friday} & \multirow{12}{*}{TP} 
                 & \multirow{2}{*}{Watson} & \multirow{2}{*}{\cmark} & 2020-06-01 \\
                 &&&                          &                        & 2020-06-05 \\\cline{4-6}
                 &&& \multirow{2}{*}{Dialogflow date-period} & \multirow{2}{*}{\xmark} & 2020-05-25 \\
                 &&&                          &                        & 2020-06-01 \\\cline{4-6}
                 &&& \multirow{2}{*}{Dialogflow date-time} & \multirow{2}{*}{\cmark} & 2020-05-25 \\
                 &&&                          &                        & 2020-06-05 \\\cline{4-6}
                 &&& \multirow{2}{*}{Spacy} & \multirow{2}{*}{\cmark} & Monday to \\
                 &&&                          &                        & Friday \\\cline{4-6}
                 &&& \multirow{2}{*}{Duckling} & \multirow{2}{*}{\cmark} & 2020-06-01 \\
                 &&&                          &                        & 2020-06-06 \\\cline{4-6}
                 &&& \multirow{2}{*}{LUIS} & \multirow{2}{*}{\xmark} & 2020-05-25 \\
                 &&&                          &                        & 2020-05-29 \\
                 \hline
        \multirow{12}{*}{2} &\multirow{12}{*}{\shortstack[l]{from next tuesday to\\ wednesday}} & \multirow{12}{*}{TP} 
                 & \multirow{2}{*}{Watson} & \multirow{2}{*}{\cmark} & 2020-06-02 \\
                 &&&                          &                        & 2020-06-03 \\\cline{4-6}
                 &&& \multirow{2}{*}{Dialogflow date-period} & \multirow{2}{*}{\xmark} & 2020-05-26 \\
                 &&&                          &                        & 2020-06-03 \\\cline{4-6}
                 &&& \multirow{2}{*}{Dialogflow date-time} & \multirow{2}{*}{\cmark} & 2020-06-02 \\
                 &&&                          &                        & 2020-06-03 \\\cline{4-6}
                 &&& \multirow{2}{*}{Spacy} & \multirow{2}{*}{\xmark} & next tuesday \\
                 &&&                          &                        & wednesday \\\cline{4-6}
                 &&& \multirow{2}{*}{Duckling} & \multirow{2}{*}{\cmark} & 2020-06-02 \\
                 &&&                          &                        & 2020-06-04 \\\cline{4-6}
                 &&& \multirow{2}{*}{LUIS} & \multirow{2}{*}{\xmark} & 2020-06-09 \\
                 &&&                          &                        & 2020-06-10 \\
                 \hline
        \multirow{12}{*}{3} &\multirow{12}{*}{\shortstack[l]{June 6 to June 7}} & \multirow{12}{*}{TP} 
                 & \multirow{2}{*}{Watson} & \multirow{2}{*}{\cmark} & 2020-06-06 \\
                 &&&                          &                        & 2020-06-07 \\\cline{4-6}
                 &&& \multirow{2}{*}{Dialogflow date-period} & \multirow{2}{*}{\cmark} & 2020-06-06 \\
                 &&&                          &                        & 2020-06-07 \\\cline{4-6}
                 &&& \multirow{2}{*}{Dialogflow date-time} & \multirow{2}{*}{\cmark} & 2020-06-06 \\
                 &&&                          &                        & 2020-06-07 \\\cline{4-6}
                 &&& \multirow{2}{*}{Spacy} & \multirow{2}{*}{\cmark} & June 6 to \\
                 &&&                          &                        & June 7 \\\cline{4-6}
                 &&& \multirow{2}{*}{Duckling} & \multirow{2}{*}{\cmark} & 2020-06-06 \\
                 &&&                          &                        & 2020-06-08 \\\cline{4-6}
                 &&& \multirow{2}{*}{LUIS} & \multirow{2}{*}{\cmark} & 2020-06-06 \\
                 &&&                          &                        & 2020-06-07 \\
                 
    \end{tabular}
    \caption{Date Span Entity Recogniton and Extraction 1} \label{tab:date_span_entity_extraction_recognition}
\end{table} \noindent

\begin{table}[h]
    \centering
    \begin{tabular}{ c | c | c | c | c | c  }
        \multirow{2}{*}{No} & \multirow{2}{*}{Input} & \multirow{2}{*}{Expected} & \multirow{2}{*}{Technology} & Recognized & Extracted \\ 
                 &&          &            & Correctly  & Value     \\ \hline \hline
        
        \multirow{12}{*}{4} &\multirow{12}{*}{\shortstack[l]{06/06 to 06/07}} & \multirow{12}{*}{TP} 
                 & \multirow{2}{*}{Watson} & \multirow{2}{*}{\cmark} & 2020-06-06 \\
                 &&&                          &                        & 2020-06-07 \\\cline{4-6}
                 &&& \multirow{2}{*}{Dialogflow date-period} & \multirow{2}{*}{\cmark} & 2020-06-06 \\
                 &&&                          &                        & 2020-06-07 \\\cline{4-6}
                 &&& \multirow{2}{*}{Dialogflow date-time} & \multirow{2}{*}{\cmark} & 2020-06-06 \\
                 &&&                          &                        & 2020-06-07 \\\cline{4-6}
                 &&& \multirow{2}{*}{Spacy} & \multirow{2}{*}{\xmark} & - \\
                 &&&                          &                        & - \\\cline{4-6}
                 &&& \multirow{2}{*}{Duckling} & \multirow{2}{*}{\cmark} & 2020-06-06 \\
                 &&&                          &                        & 2020-06-08 \\\cline{4-6}
                 &&& \multirow{2}{*}{LUIS} & \multirow{2}{*}{\cmark} & 2020-06-06 \\
                 &&&                          &                        & 2020-06-07 \\ \hline

        \multirow{12}{*}{5} &\multirow{12}{*}{\shortstack[l]{06/07 to 06/06}} & \multirow{12}{*}{TP} 
                 & \multirow{2}{*}{Watson} & \multirow{2}{*}{\xmark} & 2020-06-07 \\
                 &&&                          &                        & 2020-06-06 \\\cline{4-6}
                 &&& \multirow{2}{*}{Dialogflow date-period} & \multirow{2}{*}{\xmark} & 2020-06-07 \\
                 &&&                          &                        & 2020-06-06 \\\cline{4-6}
                 &&& \multirow{2}{*}{Dialogflow date-time} & \multirow{2}{*}{\cmark} & 2020-06-07 \\
                 &&&                          &                        & 2021-06-06 \\\cline{4-6}
                 &&& \multirow{2}{*}{Spacy} & \multirow{2}{*}{\xmark} & - \\
                 &&&                          &                        & - \\\cline{4-6}
                 &&& \multirow{2}{*}{Duckling} & \multirow{2}{*}{\xmark} & 2020-06-07 \\
                 &&&                          &                        & 2020-06-07 \\\cline{4-6}
                 &&& \multirow{2}{*}{LUIS} & \multirow{2}{*}{\xmark} & 2020-06-07 \\
                 &&&                          &                        & 2020-06-06 \\
                 \hline
        \multirow{12}{*}{6} &\multirow{12}{*}{\shortstack[l]{from tomorrow to\\06.06.2020}} & \multirow{12}{*}{TP} 
                 & \multirow{2}{*}{Watson} & \multirow{2}{*}{\cmark} & 2020-06-02 \\
                 &&&                          &                        & 2020-06-06 \\\cline{4-6}
                 &&& \multirow{2}{*}{Dialogflow date-period} & \multirow{2}{*}{\xmark} & - \\
                 &&&                          &                        & - \\\cline{4-6}
                 &&& \multirow{2}{*}{Dialogflow date-time} & \multirow{2}{*}{\xmark} & 2020-06-02 \\
                 &&&                          &                        & - \\\cline{4-6}
                 &&& \multirow{2}{*}{Spacy} & \multirow{2}{*}{\xmark} & tomorrow \\
                 &&&                          &                        & - \\\cline{4-6}
                 &&& \multirow{2}{*}{Duckling} & \multirow{2}{*}{\cmark} & 2020-06-02 \\
                 &&&                          &                        & 2020-06-07 \\\cline{4-6}
                 &&& \multirow{2}{*}{LUIS} & \multirow{2}{*}{\cmark} & 2020-06-02 \\
                 &&&                          &                        & 2020-06-06 \\
                 
    \end{tabular}
    \caption{Date Span Entity Recogniton and Extraction 2} \label{tab:date_span_entity_extraction_recognition2}
\end{table} \noindent

% precision of intent recognition
\subsection*{Intent Classification}
The main task of a chatbot is to find the intent best matching the user input.
The training utterances for the sickness intent are listed Tabel \ref{tab:sickness_utterances}
in the Appendix.
The phrases for the vacation intent can be found in Table \ref{tab:vacation_utterances}.
As an evaluation criterion, multiple parameters are used.
The confidence score represents how well the input can be classified.
Building upon the entity extraction the correct extraction of the entities in combined inputs is tested. 
A combined example is training phrase five of Table \ref{tab:sickness_utterances} 
where the entities person and date are built into the training sentence.
Furthermore, the correctly identified intents are used to calculate the f-score 
like it was done for the entities.
For the intent classification task, only date and date-spans are used which
were recognized by all relevant frameworks to ensure unbiased tests.
Spacy has no relevance for the application in terms of date and date-span
extraction and is not considered for dates in the intent classification tests.
% tests
The detailed test are listed in Table \ref{tab:sickness_intent_classification} and 
Table \ref{tab:vacation_intent_classification}.
The test evaluation is shown in Table \ref{tab:intent_classification_result}.
The tests show that Dialogflow, LUIS, and Watson have almost the same average 
confidence score. 
Dialogflow has a smaller deviation than the other three.
Dialogflow always scores above 50\% confidence and often reaches 100\%.
The scores of Watson Assistant are similar to Dialogflow with the difference 
that the confidence score was below 50\% at times.
The confidence scores of Rasa are lower than the ones of Dialogflow, LUIS, and Watson on 
average, shown in 
Table \ref{tab:intent_classification_result}.
Dialogflow and Watson managed to classify all intents correctly.
LUIS classified all vacation intents correctly.
The entity extraction of Dialogflow failed in test case 4 of 
Table \ref{tab:sickness_intent_classification} since "myself" has been extracted as a person.
This means that the person slot was set incorrectly.
LUIS extracted sick as person in test sentence one and didn't extract the person in test sentence six, shown in 
Table \ref{tab:sickness_intent_classification}.
The test sentences get more complex as the number increases.
Dialogflow and Watson were able to correctly classify all test sentences.
LUIS failed to classify test sentence seven of Table \ref{tab:sickness_intent_classification}.
Rasa failed to correctly classify the more complex sentences.
This leads to the conclusion that Dialogflow, LUIS, and Watson work better with 
complex sentences when a small amount of training data is used.
% F-score
The result in Table \ref{tab:intent_classification_result} shows that 
Dialogflow and Watson reach a perfect f-score of 1.0.
Rasa was able to reach a much higher f-score on the vacation use-case than 
on the sickness use-case.
This implies that the vacation test sentences were better classified by the system than
the sickness related sentences. 
The average confidence was highest with Dialogflow followed 
closely by LUIS and the Watson Assistant.
Compared to Dialogflow, LUIS, and Watson Assistant the average confidence
score of Rasa is a lot lower.
% German language tests
The same experiment is also interesting with German as language since the potential 
users are from a German speaking country.
The training data is listed in the Appendix in Table \ref{tab:sickness_utterances_de} and 
Table \ref{tab:vacation_utterances_de}.
The translates test sentences are listed in Table \ref{tab:sickness_intent_classification_de} 
and Table \ref{tab:vacation_intent_classification_de} with a boolean value which indicates if the 
correct intent was identified, a boolean that indicates if the entities were extracted correctly,
and the confidence score for each test case.
The finalized result is listed in \ref{tab:intent_classification_result_de} and shows 
that the average confidence score for the German test sentences is a higher than 
for the English test sentences for each technology individually.
The f-score also increased significantly which indicates that the German test sentences are 
recognized better than the English test sentences.
The ranking of the Technologies is also different from before since three technologies 
achieved perfect f-scores the ranking is based on the average confidence.
The highest confidence was achieved by Watson.
The second place goes to LUIS.
LUIS was able to increas the average confidence from about 75\% on the English tests 
to about 85\% on the German tests.
Dialogflow only reaches the third place.
The intent classification are top notch for all technologies and all can be recommended from that 
point of view.
The results in Table \ref{tab:sickness_intent_classification_de} and \ref{tab:vacation_intent_classification_de} 
show that the entity extraction results drop in the German setup with all technologies except LUIS.
Dialogflow has problems to extract the person entity correctly in German sentences although German 
has been selected in the settings. 
The system entity date-time of Dialogflow is unable to handle German date formats although 
the dialog language is set to German.
Watson doesn't offer entity annotation in training sentences when German is selected as language.
This means that a main feture is only available for English language bots.
Until the annotation of entities in training phrases is enabled for other languages Watson should only be 
used in English.
LUIS offers a person entity when English is selected but not for the German language.
The date-time entity of LUIS is able to handle German date formats. 
This means that LUIS is the only technology which achieved a satisfying result in the entity extraction category.
All in all, the intent classification result is very good for the German language tests but the entity extraction 
often fails.






\begin{table}[h]
    \centering
    \begin{tabular}{ c | l | c | c | c | c   }
        \multirow{2}{*}{No} & \multirow{2}{*}{Input} & \multirow{2}{*}{Technology} & Intent & Entity & Confid. \\ 
                 &&          & Match & Match & [\%]     \\ \hline \hline

        \multirow{4}{*}{1} & \multirow{4}{*}{sick}  
        &  Watson & \multirow{4}{*}{\cmark} & \multirow{3}{*}{-} & 41.0 \\\cline{3-3} \cline{6-6}
        && Dialogflow &  &  & 84.2 \\ \cline{3-3}\cline{6-6}
        && Rasa &  &  & 78.5 \\ \cline{3-3}\cline{5-6}
        && LUIS &  & \xmark & 99.2 \\ \hline

        \multirow{4}{*}{2} & \multirow{4}{*}{Emilia Schwarz is ill}  
        &  Watson & \multirow{4}{*}{\cmark} & \multirow{4}{*}{\cmark} & 94.0 \\ \cline{3-3}\cline{6-6}
        && Dialogflow & &  & 78.2 \\ \cline{3-3}\cline{6-6}
        && Rasa &  & & 63.1 \\ \cline{3-3}\cline{6-6}
        && LUIS &  & & 83.4 \\ \hline

        \multirow{4}{*}{3} & \multirow{4}{*}{Lea Schmitt is sick until wednesday}  
        &  Watson & \multirow{4}{*}{\cmark} & \multirow{4}{*}{\cmark} & 92.0 \\ \cline{3-3}\cline{6-6}
        && Dialogflow & & & 73.9 \\ \cline{3-3}\cline{6-6}
        && Rasa & & & 66.2 \\ \cline{3-3}\cline{6-6} 
        && LUIS &  & & 83.3 \\ \hline

        \multirow{4}{*}{4} & \multirow{4}{*}{I need to report myself as sick}  
        &  Watson & \multirow{2}{*}{\cmark} & - & 38.0 \\ \cline{3-3} \cline{5-6}
        && Dialogflow & & \xmark & 63.3 \\ \cline{3-6} % extracted myself as person
        && Rasa & \multirow{1}{*}{\xmark} & \multirow{2}{*}{-} & 9.7 \\\cline{3-4} \cline{6-6} 
        && LUIS & \cmark &  & 87.2 \\ \hline

        \multirow{4}{*}{5} & \multirow{4}{*}{\shortstack[l]{Paul Armstrong is sick and will be \\ back on Monday}}  
        &  Watson & \multirow{2}{*}{\cmark} & \multirow{4}{*}{\cmark} & 89.0 \\ \cline{3-3}\cline{6-6}
        && Dialogflow & &  & 69.6 \\ \cline{3-4}\cline{6-6}
        && Rasa & \multirow{1}{*}{\xmark} &  & 23.8 \\\cline{3-4} \cline{6-6}
        && LUIS & \cmark & & 73.1 \\ \hline

        \multirow{4}{*}{6} & \multirow{4}{*}{\shortstack[l]{sick Anna Maier 06/10}}  
        &  Watson & \multirow{4}{*}{\cmark} & \multirow{3}{*}{\cmark} & 90.0 \\ \cline{3-3}\cline{6-6}
        && Dialogflow & &  & 100.0 \\ \cline{3-3}\cline{6-6}
        && Rasa & &  & 80.5 \\\cline{3-3} \cline{5-6} 
        && LUIS &  & \xmark & 67.6 \\ \hline 

        \multirow{4}{*}{7} & \multirow{4}{*}{\shortstack[l]{I feel sickly today and won't come to work}}  
        &  Watson & \multirow{2}{*}{\cmark} & \multirow{4}{*}{\cmark} & 43.0 \\ \cline{3-3}\cline{6-6}
        && Dialogflow &  &  & 51.6 \\ \cline{3-4}\cline{6-6}
        && Rasa & \multirow{2}{*}{\xmark} &  &  4.3 \\\cline{3-3}\cline{6-6}
        && LUIS & & & 17.9 \\ \hline 

        \multirow{4}{*}{8} & \multirow{4}{*}{\shortstack[l]{Anna Lehner is feeling sick.\\She will be back on Thursday}}  
        &  Watson & \multirow{2}{*}{\cmark} & \multirow{4}{*}{\cmark} & 91.0 \\ \cline{3-3}\cline{6-6}
        && Dialogflow &  &  & 72.5 \\ \cline{3-4}\cline{6-6}
        && Rasa & \multirow{1}{*}{\xmark} &  &  26.2 \\\cline{3-4}\cline{6-6}
        && LUIS & \cmark & & 68.1 \\ 

    \end{tabular}
    \caption{Sickness Intent Classification} \label{tab:sickness_intent_classification}
\end{table} \noindent

\begin{table}[h]
    \centering
    \begin{tabular}{ c | l | c | c | c | c   }
        \multirow{2}{*}{No} & \multirow{2}{*}{Input} & \multirow{2}{*}{Technology} & Intent & Entity & Confid. \\ 
                 &&          & Match & Match & [\%]     \\ \hline \hline

        \multirow{4}{*}{1} & \multirow{4}{*}{vacation}  
        &  Watson & \multirow{4}{*}{\cmark} & \multirow{4}{*}{-} & 100.0 \\\cline{3-3} \cline{6-6}
        && Dialogflow &  &  & 100.0 \\ \cline{3-3}\cline{6-6}
        && Rasa &  &  & 91.7 \\ \cline{3-3}\cline{6-6}
        && LUIS &  &  & 97.1 \\ \hline

        \multirow{4}{*}{2} & \multirow{4}{*}{\shortstack[l]{my colleague Franz Bauer wants to\\ go on vacation}}  
        &  Watson & \multirow{4}{*}{\cmark} & \multirow{4}{*}{\cmark} & 91.0 \\\cline{3-3} \cline{6-6}
        && Dialogflow &  &  & 64.9 \\ \cline{3-3}\cline{6-6}
        && Rasa &  &  & 33.7 \\ \cline{3-3}\cline{6-6}
        && LUIS &  &  & 64.9 \\ \hline

        \multirow{4}{*}{3} & \multirow{4}{*}{\shortstack[l]{vacation Martin Huber 07/12 to\\ July 15th}}  
        &  Watson & \multirow{4}{*}{\cmark} & \multirow{4}{*}{\cmark} & 44.0 \\\cline{3-3} \cline{6-6}
        && Dialogflow &  &  & 100.0 \\ \cline{3-3}\cline{6-6}
        && Rasa &  &  & 57.4 \\ \cline{3-3}\cline{6-6}
        && LUIS &  &  & 84.6 \\ \hline

        \multirow{4}{*}{4} & \multirow{4}{*}{\shortstack[l]{holiday from 07/12 to July 15th}}  
        &  Watson & \multirow{4}{*}{\cmark} & \multirow{4}{*}{\cmark} & 94.0 \\\cline{3-3} \cline{6-6}
        && Dialogflow &  &  & 64.7 \\ \cline{3-3}\cline{6-6}
        && Rasa &  &  & 67.4 \\ \cline{3-3}\cline{6-6}
        && LUIS & &  & 87.9 \\ \hline

        \multirow{4}{*}{5} & \multirow{4}{*}{\shortstack[l]{vacation from 28th September to\\ October 4}}  
        &  Watson & \multirow{4}{*}{\cmark} & \multirow{4}{*}{\cmark} & 95.0 \\\cline{3-3} \cline{6-6}
        && Dialogflow &  &  & 100.0 \\ \cline{3-3}\cline{6-6}
        && Rasa &  &  & 55.5 \\ \cline{3-3}\cline{6-6}
        && LUIS &  &  & 88.1 \\ \hline

        \multirow{4}{*}{6} & \multirow{4}{*}{\shortstack[l]{I really need a long vacation on a \\ distant island}}  
        &  Watson & \multirow{4}{*}{\cmark} & \multirow{4}{*}{-} & 44.0 \\\cline{3-3} \cline{6-6}
        && Dialogflow &  &  & 79.0 \\ \cline{3-3}\cline{6-6}
        && Rasa &  &  & 35.0 \\ \cline{3-3}\cline{6-6}
        && LUIS &  &  & 92.3 \\ \hline

        \multirow{4}{*}{7} & \multirow{4}{*}{\shortstack[l]{colleague Helmut Kerschbaum \\ requested holidays}}  
        &  Watson & \multirow{4}{*}{\cmark} & \multirow{4}{*}{\cmark} & 93.0 \\ \cline{3-3}\cline{6-6}
        && Dialogflow & & & 61.7 \\ \cline{3-3}\cline{6-6}
        && Rasa & &  & 46.6 \\\cline{3-3}\cline{6-6}
        && LUIS &  & & 41.9 \\ \hline

        \multirow{4}{*}{8} & \multirow{4}{*}{\shortstack[l]{Colleague Helmut Kerschbaum \\requested holidays. He will be \\unavailable from 09/10 to 09/24}}  
        &  Watson & \multirow{2}{*}{\cmark} & \multirow{4}{*}{\cmark} & 45.0 \\ \cline{3-3}\cline{6-6}
        && Dialogflow & & & 70.2 \\ \cline{3-4}\cline{6-6}
        && Rasa & \multirow{1}{*}{\xmark} &  & 21.6 \\\cline{3-4}\cline{6-6}
        && LUIS &  \cmark & & 70.8 \\ 

    \end{tabular}
    \caption{Vacation Intent Classification} \label{tab:vacation_intent_classification}
\end{table} \noindent

\begin{table}[h]
    \centering
    \begin{tabular}{ l | l | c | c | c | c | c | c | c | c }
        \multirow{2}{*}{Framework} & \multirow{2}{*}{Tested} & \multirow{2}{*}{TP} & \multirow{2}{*}{FN} & \multirow{2}{*}{p} & \multirow{2}{*}{r} & \multirow{2}{*}{f-score} & Confid. & Confid. & std.\\ 
                 &&          &            &   & & & avg. [\%] & Median & dev.  \\ \hline \hline
        \multirow{3}{*}{Watson} 
        & sickness & 8 & 0& 1.0& 1.0& 1.0&72.3 & 89.5 & 26.2\\
        & vacation & 8 & 0& 1.0& 1.0& 1.0&75.8 & 92.0 & 26.1 \\
        & sum      &16 & 0& 1.0& 1.0& 1.0&74.0 &  90.5 & 25.4 \\ \hline
        
        \multirow{3}{*}{Dialogflow} 
        & sickness & 8& 0& 1.0& 1.0& 1.0&74.2&73.2 & 14.3\\
        & vacation & 8& 0& 1.0& 1.0& 1.0&80.1 & 74.6 & 17.3\\
        & sum      &16& 0& 1.0& 1.0& 1.0&77.1 & 73.2 & 15.6\\ \hline

        \multirow{3}{*}{Rasa} 
        & sickness & 4& 4& 1.0& 0.5  & 0.67& 44.0 &44.7 & 31.3\\
        & vacation & 7& 1& 1.0& 0.88& 0.93& 51.1&51.1& 22.1\\
        & sum      &11& 5& 1.0& 0.69& 0.82& 47.6&51.1 & 26.4\\ \hline

        \multirow{3}{*}{LUIS} 
        & sickness & 7& 1& 1.0& 0.88  & 0.93& 72.5&78.2& 24.5\\
        & vacation & 8& 0& 1.0& 1.0& 1.0& 78.5&86.3 & 18.3\\
        & sum      & 15& 1& 1.0& 0.94& 0.97& 75.5&83.4 & 21.1\\
    \end{tabular}
    \caption{Intent Classification Result} \label{tab:intent_classification_result}
\end{table} \noindent

\begin{table}[h]
    \centering
    \begin{tabular}{ l | l | c | c | c | c | c | c | c | c }
        \multirow{2}{*}{Framework} & \multirow{2}{*}{Tested} & \multirow{2}{*}{TP} & \multirow{2}{*}{FN} & \multirow{2}{*}{p} & \multirow{2}{*}{r} & \multirow{2}{*}{f-score} & Confid. & Confid. & std.\\ 
                 &&          &            &   & & & avg. [\%] & Median & dev.  \\ \hline \hline
        \multirow{3}{*}{Watson} 
        & sickness & 8 & 0& 1.0& 1.0& 1.0&94.1& 94.0& 1.88\\
        & vacation & 8 & 0& 1.0& 1.0& 1.0&96.4& 96.5&1.99\\
        & sum      &16 & 0& 1.0& 1.0& 1.0&95.3& 95.5&2.21 \\ \hline
        
        \multirow{3}{*}{Dialogflow} 
        & sickness & 8& 0& 1.0& 1.0& 1.0&80.8& 78.6&19.1\\
        & vacation & 8& 0& 1.0& 1.0& 1.0&89.5& 100& 16.5\\
        & sum      &16& 0& 1.0& 1.0& 1.0&85.1&90.3 &17.8\\ \hline

        \multirow{3}{*}{Rasa} 
        & sickness & 7& 1& 1.0& 0.88  & 0.93& 54.0& 44.2&20.2\\
        & vacation & 8& 0& 1.0& 1.0& 1.0& 58.3& 57.0&10.9\\
        & sum      &15& 1& 1.0& 0.94& 0.97& 56.2&53.6 &15.9\\ \hline

        \multirow{3}{*}{LUIS} 
        & sickness & 8& 0& 1.0& 1.0 & 1.0& 79.6& 77.5&15.1\\
        & vacation & 8& 0& 1.0& 1.0 & 1.0& 91.0& 90.9&5.01\\
        & sum      & 16&0& 1.0& 1.0 & 1.0& 85.3& 89.7&12.3\\
    \end{tabular}
    \caption{Intent Classification Result German} \label{tab:intent_classification_result_de}
\end{table} \noindent

% training capabilities
\subsection*{Training Capabilities}
One major factor for chatbot technologies is the training of intents, entities, and stories/dialogs.
The development of new intents and entities works the same for all four technologies.
An intent is trained wiht utterances.
An entity is created either with examples or annotations.
The training sentences for the intents can contain entities.
They can be marked as enitties by annotating them.
Then the values are used to train the entity.
This porcess is the same for all four technologies, hence there is no 
difference in the creation process of intents and entities.
The creation of stories/dialogs only works for chatbot technologies.
LUIS won't be part of the story evaluation.
The creation of stories is done with dialog nodes in Dialogflow and Watson while
Rasa uses stories.
A dialog node represents a recognized intent and defines the slots, actions, and responses regarding 
the intent.
In the UI follow up dialog nodes can be creted to build a structure where one intent follows after another.
Rasa uses stories which also defined the recognized intent, the slots, actions and responses.
The real difference is that the stories of Rasa can be structured in any imaginable way while the 
dialog nodes follow a fixed structure.
The second big difference is that live training should be used with Rasa since the structuring of the story files can get 
rather complicated.
Neither Dialogflow nor Watson Assistant offer live training.
With live training a developer starts a conversation with the bot and goes through the process.
Everytime the bot makes a mistake the developer can correct the bot.
The second option is to define the conversation step-by-step from scratch where the developer tells the 
bot what to do.
When the end of a story is reached it is stored and used for training.
The training capabilites of Rasa exceed the ones of Dialogflow and Watson Assistant and is ranked first. 
There is no difference between Dialogflow and Watson Assistant when the dialog structure creation is considered.


% unique capabilities
% speech recognition capabilities
\subsection*{Speech Recognition Capabilities}
For the further development which is based on the results of this thesis speech recognition 
is an option. 
The result of this evaluation is true or false.
Frameworks which support this feature can be used with speech input instead of text.
Either the tchnology supports speech recognition or it doesn't. 
Rasa doesn't support speech functionality itself but it can be used in combination with 
speech-to-text and text-to-speech frameworks like Alexa.
Dialogflow offers audio in- and oputput.
The speech-to-text functonality of Dialoflow costs \$0.0065 per 15 seconds of audio.
The IBM Watson Assistant also offers speech functionalities through Watson Speech to 
Text and Watson Text to Speech.
The Watson text-to-speech service costs between \$0.01 and \$0.02 per minute which results 
in \$0.0025 to \$0.005 per 15 seconds.
The speech-to-text service costs \$0.02 per 1,000 chars which results in 
\$20 per million chars.
Microsoft offers the Speech Services for the speech-to-text and text-to-speech functionality.
The Speech service costs \$1 per hour of speech which results in \$0.00416 per 15 seconds which is 
less than the costs for Dialoflow.
The costs for text-to-speech are the same for the Speech Services and Dialogflow with a 
rate of \$4 per one million characters.
The speech-to-text services are on a similar pricing level but the text-to-speech service of 
Watson costs five times as much as the Speech Services or Dialogflow.
The only technology which doesn't support speech recognition is Rasa.
The other three technologies support the speech-to-text and text-to-speech features 
which are required for speech processing.   


\subsection*{Framework Concepts}
The user interfaces of the cloud frameworks are similar.
The interfaces are shown in Figure \ref{fig:dialogflow_interface},
\ref{fig:watson_interface}, and \ref{fig:luis_interface}.
All three offer intents and entities.
They are all structured similarly with the menu to the left.
The information for the bots is always entered as text.
Intents are always built with training phrases.
All offer prebuilt system entities.
Dialogflow has the most prebuilt entities, followed by LUIS,
then comes Rasa, and Watson Assistant offers the fewest entities.
For the sickness and vacation use-case the entities of 
Dialogflow suit best because no entity has to be defined by 
the developer. 
Watson offers a person system entity in English but the entity
is marked as deprecated.
It offers a date entity but no date span entity.
This means with Watson Assistant two entities have to be 
modeled for the use-cases.
LUIS offers a date-time entity which can be used for dates and 
date spans but it's not working as reliable as the one from 
Dialogflow.
It did not identify "22-06 to 30-06" correctly
which is no problem for Dialogflow.



% usability framework
\subsection*{Usability of the Framework}



% customization possibilities
\subsection*{Customization Possibilities}
A major part of chatbot frameworks is the pipeline since it defines how the 
training of the intents and entities is done.
The cloud chatbot technologies don't show the pipeline they are using nor can you 
adjust it.
Rasa is different in that regard since the pipeline can be adjusted freely.
Furthermore, the technologies like Spacy show information about the components 
online.
Spacy can be used with different models like small, medium, and large training sets 
for the predefined entities when English is the target language.
The pipeline used for this thesis is shown in Figure \ref{lst:spacy_pipeline_detail}.
It's even possible to include custom components into the pipeline of Rasa.
This means that Rasa is the only technology of the four where the pipeline is visible 
and it offers customization on top of that.
The other three technologies don't offer mechanisms to adjust the pipeline.
It's possible to use external services instead of adjusting the pipeline by 
moving the enity extraction to the webhook where necessary.
Since especially Dialogflow and LUIS offer a huge amount of predefined entities 
the use of other entity extraction tools might not be necessart at all.
It's different for Watson Assistant since it has a small amount of predefined entities.

