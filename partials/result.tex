
\section{Framework Comparison}

\subsection*{Portability}
An important factor for software systems is the portability.
Portability is defined in ISO 25010 \cite{iso25010}.
As measurment criteria the number of providers which support the technology is defined.
A subfactor is how tied to a provider a technology is.
When it comes to the chatbot cloud services (LUIS, Dialogflow, IBM Watson Assistant)
the provider can't be changed.
All three services run in the cloud of the provider an nowhere else.
The data processing is done in the cloud and can't be done local or on a company server.
This means that the three cloud services are not portable at all and 
the provider can't be changed.
With Rasa things are different.
It is a local solution which means it can run on a local machine and a company server.
The data processing is also done locally and never goes into the cloud of a provider.
Docker images for Rasa are provided too.
It's possible to run Docker containers in the clouds of all big providers.
This means Rasa can run everywhere where Docker or a Pyhton environment are possible 
and the provider can be changed easily.
Another aspect of portability is the number of devices where the chatbot can be developed
from.
The cloud frameworks are accessible via the internet and can be used 
and developed from every device which has internet connection and a
browser.
Rasa can only be developed on machines which have access to the 
source files.
It is per default not accessible via the internet.
However, it can be deployed in a cloud environment.
Then it can be used like the cloud services but it 
still can't be developed easily on any device.
A third aspect of portability is the migration from one technology to another.
Rasa offers an import functionality (beta version) which allows The
import of Dialogflow, LUIS, Wit.ai and IBM Watson chatbots.
In general, all chatbot frameworks used use JSON format to store
the training data.
But evert technology saves information differently and it's not 
easy to migrate an existing chatbot to another technology.

% project complexity
\subsection*{Project Setup Complexity}

% development complexity
\subsection*{Development Complexity}
% price
\subsection*{Pricing} \label{sec:pricing}
Dialogflow can be used by companies for free.
The limits for the free tier are 180 text requests per minute.
With Essential and Plus 600 requests per minute are possible.
Rasa is a free to use open source tool and costs nothing.
There is an enterprise edition with some benefits but the price
is calculated individually.
Watson Assistant offer a free tier with up to 1.000 users per month, 
10.000 messages per month.
A user is a person which interacts with the bot at least once.
The Plus edition costs \$120 for 1.000 users per month with
no message or user limit.
The pricings and editions of the framework are shown in Table \ref{tab:pricing}.
The price of Dialogflow is based on the number of requests,
IBM offers a fixed number of requests and users, and with Rasa the developer has
to take care of the number of users and requests in soft- and hardware.
In the free tier Dialogflow provides the best value because there is message or user limit.
In the priced tier Watson costs \$0.12 per user per month (\$120 for 1.000 users per month).
Dialogflow charges \$0.002 per request.
Watsons cost of \$0.12 per user per month equals 60 requests per user per month with Dialogflow.


\begin{table}[H]
    \centering
    \begin{tabular}{ l | l | l }
        Framework & Edition & Price \\ \hline \hline
        \multirow{3}{*}{Dialogflow} & Standard & Free \\
        & EE Essentials &  \$0.002 per request\\
        & EE Plus & \$0.004 per request \\ \hline

        \multirow{3}{*}{Watson Assistant} & Lite & Free \\
        & Plus &  \$120 for 1000 users/month\\
        & Premium & Individual \\ \hline

        \multirow{3}{*}{Rasa} & Rasa Open Source & Free \\
        & Rasa X &  Free\\
        & Rasa Enterprise & Individual \\ \hline
    \end{tabular}
    \caption{Pricing of Frameworks \cite{rasa, dialogflow, watsonassistant}} \label{tab:pricing}
\end{table} \noindent


% customization possibilities
\subsection*{Customization Possibilities}
% needed knowledge
\subsection*{Learnability (Needed Knowledge)}
The easiest of the chatbot frameworks is Dialogflow.
Compared to Rasa the setup is easy.
The UI is minimalistic and easy to understand.
It's easier to develop a chatbot wiht Dialogflow than wiht Watson Assistant or Rasa.
The Watson Assistants UI is intransparent when comapred to Dialogflow and a developer 
may need more time to orient.
Rasa requires the most time to develop a chatbot and to get everything ready.
% deployment
\subsection*{Deployment Complexity}
% precision of intent recognition
\subsection*{Intent Classification Precision}
% TP, FP, TN, FN
% response times
\subsection*{Response Times}
% precision entity recognition/extraction
\subsection*{Entity Extraction Capabilities}
% predefined types/entities
\subsection*{Predefined Entities}
% training capabilities
\subsection*{Training Capabilities}
% live training 
% dialogflow
% integration
\subsection*{Integration}
% exception/error handling
\subsection*{Error Handling}
% transfer of knowledge between technologies
\subsection*{Transfer of Knowledge}
% usability framework
\subsection*{Usability of the Framework}
% usability frontend
\subsection*{Frontend}
% unique capabilities
% speech recognition capabilities
\subsection*{Speech Recognition Capabilities}

\subsection*{Framework Concepts}
The user interfaces of the cloud frameworks are similar.
The interfaces are shown in Figure \ref{fig:dialogflow_interface},
\ref{fig:watson_interface}, and \ref{fig:luis_interface}.
All three offer intents and entities.
They are all structured similar with the menu to the left.
The information for the bots is always entered as text.
Intents are always built with training phrases.
All offer prebuilt system entities.
LUIS has the most prebuilt entities, Dialogflow is in the middle,
and Watson Assistant offers the fewest entities.
For the sickness and vacation use-case the entities of 
Dialogflow suit best because no entity has to be defined by 
the developer. 
Watson offers a person system entity in English but the entity
is marked as deprecated.
It offers a date entity but no date span entity.
This means with Watson Assistant two entities have to be 
modeled for the use-cases.
LUIS offers a datetime entity which can be used for dates and 
date spans but it's not working as reliable as the one from 
Dialogflow.
It did not identify "22-06 to 30-06" correctly
which is no problem for Dialogflow.



