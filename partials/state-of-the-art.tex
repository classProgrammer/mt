% Hybrid coding networks
\citet{williams2017hybrid} and \citet{bordes2016learning} state that the two major chatbot types are goal-oriented and end-to-end systems.
% Kane 2016 role of chatbots in teaching and learning
\citet{kane2016role} defines two different categories of chatbots.
They can either be classified as a local standalone application or a web-based chatbot \cite{kane2016role}.
The types/classes of chatbots from \citet{williams2017hybrid, bordes2016learning} and \citet{kane2016role} 
don't contradict they just look at chatbots from a different perspective.
From the design perspective the conversation type of a chatbot is either goal-oriented or end-to-end.
From the development perspective the used framework is either web-based or local.
Goal-oriented or end-to-end is a general statement about the kind of base problem 
which the chatbot needs to solve.
Web-based or local has an influence on where the chatbot is developed.
Rasa is one example for a local standalone application chatbot whereas LUIS, 
Dialogflow, and Watson Assistant are cloud-based/web-based chatbot frameworks.
Rasa, Dialogflow, LUIS, and Watson Assistant are all capable of goal-oriented dialog.

The first major type are end-to-end systems.
% ---------- END-TO-END SYSTEMS ---------- %
% evaluate chatbots Shawar
\citet{evaluateChatbotsShawar2007} focus on the application of chat systems in different languages and the evaluation of the system.
The evaluation of chatbot systems needs to be adapted to the application and the user needs \cite{evaluateChatbotsShawar2007}.
The best way to evaluate a chatbot is to check if the specific service or task can be achieved and the evaluation 
needs to be adjusted to the use-case \cite{evaluateChatbotsShawar2007}.
Each component of a chatbot system can be tested individually and the whole system can be tested based on user 
satisfaction and acceptance \cite{evaluateChatbotsShawar2007}.
The whole system is evaluated through feedback by users in a test setting \cite{evaluateChatbotsShawar2007}.
This feedback is then used to improve the system \cite{evaluateChatbotsShawar2007}. 
In \citet{evaluateChatbotsShawar2007} a chatbot for frequently asked questions (FAQ) has been built 
which competed against the classical Google search.
The participants had to answer a few questions like how many results the system found and which system they prefer 
and why \cite{evaluateChatbotsShawar2007}. 
The result of the user study was that the FAQ chatbot found an answer more often than Google \cite{evaluateChatbotsShawar2007}.
47\% of the users taking part in the study preferred the custom solution and 11\% preferred Google \cite{evaluateChatbotsShawar2007}.
The major reason why the users preferred the FAQ chat was because the chat can often give direct answers and return 
fewer links on average which saves browsing time \cite{evaluateChatbotsShawar2007}. 
The users preferred Google because it's familiar and it can give different answers when the input query is adjusted 
slightly \cite{evaluateChatbotsShawar2007}.
This shows that custom-built solutions can outperform classical approaches in small areas.

% Bordes 2016 learning goal-oriented dialog
\citet{bordes2016learning} analyses the strengths and weaknesses of end-to-end dialog systems in goal-oriented settings.
The end-to-end approaches have recently shown promising results with chit-chat/general conversation with a user \cite{bordes2016learning}.
However, The most useful application areas for chatbots are either goal-oriented or transactional settings \cite{bordes2016learning}.
A classical dialog system uses slot-filling to collect information from the conversation \cite{bordes2016learning}.
Slot-filling is reliable but has limited scalability \cite{bordes2016learning}.
The slots are predefined and hence don't scale well becasue they can be very different for each problem. 
End-to-end systems scale well because they learn directly from conversations and make 
no assumptions regarding the domain or dialog structure \cite{bordes2016learning}. 
The goal-oriented example used by \citet{bordes2016learning} is the classic restaurant reservation bot.
The main question is if the end-to-end system performs well on the classical goal-oriented restaurant 
reservation task.
The restaurant reservation is a domain where goal-oriented systems perform well and is the ideal candidate
to determine the strengths and weaknesses of end-to-end systems in comparison to goal-oriented systems \cite{bordes2016learning}.
The result of \citet{bordes2016learning} show that end-to-end systems are not yet ready to replace goal-oriented 
systems and have to improve before they can perform reliably in goal-oriented settings. 
\citet{bordes2016learning} compared the chatbot types based on the number of correctly identified requests (true positives)
and the number of dialogs where every request was identified correctly.
Multiple conversations were used to test the system and the end-to-end system never identified 
all requests of the conversation correctly \cite{bordes2016learning}.
The request identification as such worked well \cite{bordes2016learning}.
No conversation went smooth since at least one request was identified incorrectly.
In other words, each user had at least one problem in the conversation.
This would lead to a bad user experience. 
In summary, the most important statement of \citet{bordes2016learning} is that the current end-to-end systems are
not performing reliably enough in goal-oriented settings.

\citet{williams2017hybrid} intorduces an approch which combines end-to-end dialog with domain specific knowledge.
A task-oriented system helps the user to achieve a task using natural language \cite{williams2017hybrid}.
A restaurant reservation is a task-oriented operation which requires domain-specific knowledge \cite{williams2017hybrid}.
The domain-specific knowledge in a restaurant setting can be the restaurant name and the number of people for 
the table reservation for instance. 
This information needs to be collected from the conversation.
End-to-end systems lack a mechanism to inject such domain-specific information \cite{williams2017hybrid}.
There is also no mechanism for constraints in goal-oriented systems \cite{williams2017hybrid}.
In a banking app the user needs to be logged in (constraint) before the account information can be retrieved \cite{williams2017hybrid}.
A programmer can code such constraints with a few lines of code but lots of training data needs to be provided 
to a system to learn such a mechanism on its own \cite{williams2017hybrid}.
The hybrid-coding-networks (HCNs) introduced in \citet{williams2017hybrid} extend the open-ended approach 
with domain-specific knowledge. 
The domain specific knowledge is provided by the programmer and requires more development effort \cite{williams2017hybrid}.
Since the domain-specific knowledge is provided by the programmer the system doesn't need to learn 
these parts and less training data is required in comparison to existing end-to-end apporaches \cite{williams2017hybrid}.
A large amount of training data can be retrieved from real dialogs \cite{williams2017hybrid}.
The experiments show that the HCN approach performs on an equal level with 
existing end-to-end approaches in end-to-end settings and needs less training data for 
the same performance \cite{williams2017hybrid}. 
The HCN system also performed better in the task-specific setting than the existing end-to-end approaches \cite{williams2017hybrid}.
HCNs are a mixture of end-to-end and goal-oriented systems.
The developer has more control than with the classical end-to-end approach but 
it also requires more development effort to create a chatbot with HCNs \cite{williams2017hybrid}.
The main benefits are the usage of domain specific information in end-to-end dialog 
and the reduced amount of training data required.
Other references focus on end-to-end and/or goal-oriented chatbots whereas \citet{williams2017hybrid} combined them 
to get the best of both worlds.
\citet{williams2017hybrid} does not compare the HCN approach to a goal-oriented approach like \citet{bordes2016learning} did
for regual end-to-end systems.
Hence, no statement can be made if HCNs could replace goal-oriented systems or not.

% ---------- EVAL ---------- %
In \citet{evaluateChatbotsShawar2007} testing focus lies on the user side whereas the tesing focus of 
\citet{braunEvaluatingNLU} lies on the NLU capabilities.
The NLU capabilities are important for the developers of the systems before a chatbot
is developed and help to choose the best framework for development.
The users feedback is a valuable source of information for the actual product and
is important long after the framework has been choosen.
These are two very different views on the performance of a chatbot in two different phases of a project.
\citet{bordes2016learning} focues on end-to-end systems like \citet{evaluateChatbotsShawar2007, williams2017hybrid}
but uses a goal-oriented setting for the comparison.
The important information gained by \citet{bordes2016learning} is that currently end-to-end systems 
are not reliable enough for goal-oriented tasks.
This leads to the conclusion that a goal-oriented technology should be choosen for goal-oriented 
projects.
\citet{williams2017hybrid} is trying to combine the end-to-end and goal-oriented approaches to 
create a new type of system which can do goal-oriented dialog better than current end-to-end
approaches. 
The result of \citet{williams2017hybrid} shows that the HCN system is better in goal-oriented 
settings than current end-to-end systems but doesn't compare the system to goal-oriented technologies.
% ---------- END END-TO-END SYSTEMS---------- %

% ---------- BOTH SYSTEMS---------- %
% singh building: building an enterprise chatbot
\citet{singhbuilding} states that the conversation type of a chatbot can be categorized as a general conversation 
which is about a broad generic subject or a specific conversation about one product or service.
An example of a general conversation is e.g. when a customer walks into a bank and starts talking to an employee \cite{singhbuilding}.
In this example, we have no idea what the person wants or who the person is \cite{singhbuilding}.
Popular examples for chatbots capable of general conversation are Google Home, Siri, and Amazon Alexa \cite{singhbuilding}.
An example of a specific conversation is the refund desk in a store \cite{singhbuilding}. 
Specific rules apply to the refund process and specific information is required from the customer \cite{singhbuilding}.
The domain could be named refund and the specific rules and information are part of the domain. 
The customer can't get any other information than refund related information at the refund desk \cite{singhbuilding}. 
The outcome of the conversation can be one of the predefined outcomes, the fallback, or a conversation end \cite{singhbuilding}. 
A specific conversation ends as soon as the underlying task is achieved or a conversation end is reached \cite{singhbuilding}. 
In general, specific conversations are easier to predict and are handled with higher accuracy \cite{singhbuilding}.
Chatbot systems for specific conversations communicate with information \cite{singhbuilding}. 
Nowadays, chatbots are AI-driven.
The core component of AI-driven chatbots is the NLP engine which takes care of data extraction from natural language \cite{singhbuilding}.
The extracted information is used to determine the next steps \cite{singhbuilding}.
There is no difference in functionality between a regular application and a chatbot in terms of functionality \cite{singhbuilding}. 
The difference is that a chatbot uses conversation whereas a regular app is a self-service application \cite{singhbuilding}. 
Conversational training data for chatbots can be acquired from lots of sources like emails, phone calls, chats, and social media \cite{singhbuilding}.
Companies use chatbots because they can save a lot of money and because they offer a good customer experience \cite{singhbuilding}.
Especially in areas with personal data the usage of cloud chatbot solutions like Dialogflow, Alexa, and Watson 
is problematic because all the conversation data is sent to and stored at the providers servers \cite{singhbuilding}.
Because of high-security requirements, an in-house/local chatbot has been built in \citet{singhbuilding}.
The first step to create a chatbot system is to define a conversational flow \cite{singhbuilding}.
The conversational flow is a decision tree which describes all events, decision, and outcomes in a conversation \cite{singhbuilding}.
\citet{singhbuilding} also gives an overview of the basics of Microsoft Bot (LUIS), Rasa, and Google Dialogflow.
% ---------- BOTH SYSTEMS---------- %

The second major type are goal-oriented systems.
% ---------- GOAL-ORIENTED SYSTEMS ---------- %
% Braun evaluating NLU
\citet{braunEvaluatingNLU} states that modern conversational agents can be built without programming 
knowledge because of natural language understanding services (NLU) \cite{braunEvaluatingNLU}. 
The recent advancements in machine learning (ML) and NLU and the popularity of messenger platforms 
has led to huge progress in recent years \cite{braunEvaluatingNLU}. 
People are familiar with messengers because many use them on a daily basis.
Hence, chatting has become a natural way of communication.
A big question is how chatbots can be evaluated.
Recent publications have discussed the use of NLU service but not why 
one service has been chosen over another \cite{braunEvaluatingNLU}.
In general, The architecture of a chatbot consists of three elements \cite{braunEvaluatingNLU}. 
The request needs to be interpreted, the response needs to be retrieved and the response message
needs to be generated \cite{braunEvaluatingNLU}. 
The purpose of these services is the extraction of information from natural language \cite{braunEvaluatingNLU}. 
Popular NLU services are the cloud services LUIS, Watson Conversation, Dialogflow/API.ai, and RASA as an open-source alternative \cite{braunEvaluatingNLU}. 
The cloud-based solutions have advantages when it comes to hosting and scalability and Rasa has advantages
when adaptability and data control are needed \cite{braunEvaluatingNLU}. 
All three NLU services share the same basic concpets of intents, entities and batch import in JSON format \cite{braunEvaluatingNLU}.
The cloud-based services are secretive when it comes to the ML algorithms and the initial training data \cite{braunEvaluatingNLU}. 
The exception is Rasa where the ML backend can be chosen by the developer \cite{braunEvaluatingNLU}. 
For the evaluation of the NLU services training data from a production chatbot and two other training sets 
from SatckExchange were used \cite{braunEvaluatingNLU}. 
The NLU capabilities of the frameworks were compared.
The comaprison included the services LUIS, Rasa, Dialogflow and Watson Conversation \cite{braunEvaluatingNLU}. 
The same training data was used for all services \cite{braunEvaluatingNLU}. 
This is important to get a comparable result.
The evaluation of the services NLU capability is based on the true and false positives and negatives, 
the recall, precision, and F-score \cite{braunEvaluatingNLU}. 
The better the F-score is the better the NLU service has performed \cite{braunEvaluatingNLU}. 
The evaluation of NLU services is only a snapshot since the services are continuously improving \cite{braunEvaluatingNLU}. 
LUIS performed best on all datasets but in the area of chatbots Rasa and LUIS performed on an equal level \cite{braunEvaluatingNLU}.
Rasa can be customized further than LUIS and could perform better than LUIS through customization \cite{braunEvaluatingNLU}. 
The domain had little to no influence on the ranking of the NLU performance \cite{braunEvaluatingNLU}.
In other words, a good NLU system performs good independent of the domain.
If the training data is sparse then there is no significant difference between the services \cite{braunEvaluatingNLU}. 
Before using an NLU service different services should be tested with domain-specific data \cite{braunEvaluatingNLU}.

% Dutta 2017 developing
In \citet{dutta2017developing} a chatbot is developed that assists high school students with learning general 
knowledge subjects and analyses the impact the chatbot had on learning.
AI-based chatbots are used for banking systems, customer services, and education \cite{dutta2017developing}.
Popular chatbot platforms are Dialogflow, LUIS, Wit.ai amd Pandorabots \cite{dutta2017developing}.
The developed chatbot is a web-based education solution using natural language processing (NLP) techniques
to answer questions \cite{dutta2017developing}.
It is also able to participate in small talk \cite{dutta2017developing}.
The chatbot systems are evaluated based on their NLP performance and the available features of the platform 
\cite{dutta2017developing}. 
All platforms are trained with the same knowledge base for comparison \cite{dutta2017developing}. 
LUIS and Wit.ai performed slightly better in the NLP part than Dialogflow \cite{dutta2017developing}.
Dialogflow offers the concept of follow-up intents which is an important feature for the development 
of sub-tasks and was chosen for development over LUIS and Wit.hai \cite{dutta2017developing}.
To create the illusion of talking to a human being some small talk is implemented \cite{dutta2017developing}. 
This can motivate learners and might increase the interest in the topic \cite{dutta2017developing}.
\citet{dutta2017developing} uses goal-oriented technologies like \citet{braunEvaluatingNLU} and 
both compare the technologies based on their NLU capabilities.
Additionally, \citet{dutta2017developing} also focuses on the available features of the 
platforms and selects Dialogflow over the other technologies because 
of these features.
\citet{dutta2017developing} also touches psychological aspects of chatbot systems 
like motivation of users through smalltalk. 
\citet{GO2019304, brandtzaeg2018chatbots,folstad2017chatbots} go deeper into the psychological aspects of chatbot systems.

% Gregori 2017 evaluation
\citet{gregori2017evaluation} evaluated the NLU capabilities of Wit.ai, Luis, Api.ai/Dialogflow, and Amazon Lex\cite{lexconversational}
by the confidence score. 
The higher the confidence the better the user request matched the training data \cite{gregori2017evaluation}.
Each tool was trained with the same data and was tested with the same questions to ensure fair conditions \cite{gregori2017evaluation}.
LUIS, Wit.ai, and API.ai/Dialogflow performed on an equal level in terms of intent classification in \citet{gregori2017evaluation}.
\citet{gregori2017evaluation} chose Api.ai/Dialogflow for the development of a prototype.

% IEEE: pharmacy bot
In \citet{pharmacybot} a chatbot is built to aid customers with questions regarding medication for a 
pharmacy company using IBM Watson.
The bot can provide medications for an illness, give information about a specific medicine, and can 
give information on the medication intake \cite{pharmacybot}.
The bot is a domain specific bot since it answers questions to medication only.

% In bot we trust:
Important performance metrics for chatbots are the conversation length and structure, the ability to 
provide personalized communication, and the number of conversation steps \cite{PRZEGALINSKA2019785}.
The general trend is to keep the conversations short \cite{PRZEGALINSKA2019785}.
Personalized communication is used when recommendations or tips are provided for the user based on the 
information related to the user \cite{PRZEGALINSKA2019785}.
Retail chatbots need a larger amount of conversation steps to provide information and recommendations 
and hold the user's attention \cite{PRZEGALINSKA2019785}.
The measurement criteria for chatbots vary depending on the domain of the bot.

% Luis 2015 Williams
\citet{luis2015williams} gives an overview of LUIS the state-of-the-art language understanding service of Microsoft.
Historically, language understanding could be implemented via machine learning or handcrafted rules \cite{luis2015williams}.
The ML model approach is robust but requires expensive expertise \cite{luis2015williams}. 
In general, software developers can build language understanding without assistance of a framework
with handcrafted rules \cite{luis2015williams}. 
But systems with handcrafted rules don't scale well \cite{luis2015williams}.
With LUIS developers don't need machine learning knowledge to build language understanding models 
nor do they need to write handcrafted rules \cite{luis2015williams}.
LUIS allows regular developers without ML expertise to develop cloud-based, domain-specific language 
understanding models \cite{luis2015williams}.
Developers need to understand the concepts intent, entity, and utterance to work with LUIS \cite{luis2015williams}.
Developers can create custom entities and use existing entities like location, date, and time.
The communication with LUIS is done through an HTTP endpoint in JSON format \cite{luis2015williams}. 
Dialogflow, Rasa, and Watson Assistant also uses an HTTP endpoint for communication and JSON format for the messages. 


% rasa bocklisch 2017
\citet{rasabocklisch2017} describes the architecutre and function of Rasa.
Rasa is an open-source tool for natural language understanding and dialog management for developers \cite{rasabocklisch2017}.
Rasa uses the concepts of intent and entity \cite{rasabocklisch2017}. 
The cloud chatbot technologies have these two concepts in common with Rasa.
The state is represented by slots and the events which led to the current state \cite{rasabocklisch2017}. 
This information is stored in the tracker \cite{rasabocklisch2017}.
An action has access to the tracker and determines the next step to take \cite{rasabocklisch2017}. 
An action can be something simple like an utterance or something complex like a function which is executed \cite{rasabocklisch2017}.
Rasa consists of a natural language and a dialog component that can be accessed via an HTTP API \cite{rasabocklisch2017}.
The training data for Rasa can be specified in JSON or markdown format \cite{rasabocklisch2017}.
Rasa supports live training of the system where the developer can correct the bot while communicating with it \cite{rasabocklisch2017}.
Live training is an effective way to create training data for plausible conversations \cite{rasabocklisch2017}.
With live training there is a higher possibility to create a complete and natural conversation because the 
developer is forced to go through the conversation step by step.
\citet{braunEvaluatingNLU} showed that Rasa performs on an equal level with tools like LUIS.

% ---------- EVAL ---------- %
\citet{braunEvaluatingNLU} compares some NLU frameworks regarding their natural language capabilities.
The compared frameworks are suitable for task-specific problems.
\citet{evaluateChatbotsShawar2007,bordes2016learning,williams2017hybrid} 
focus on end-to-end systems, \citet{braunEvaluatingNLU, dutta2017developing,luis2015williams,rasabocklisch2017,pharmacybot,gregori2017evaluation}
focus on task-oriented systems, and \citet{singhbuilding} is about both.
\citet{singhbuilding} gives a general overview of the chatbot types, gives develoment adivce, and introduces 
popular chatbot technologies and their basics.
\citet{braunEvaluatingNLU} gives an overview of the architecture of chatbot systems and 
evaluates the NLU capabilites of chatbot frameworks.
\citet{gregori2017evaluation} and \citet{dutta2017developing} also evaluate the NLU capabilities.
When frameworks are compared for their NLU capabilities the same training data needs to be used 
for all tested frameworks like \citet{braunEvaluatingNLU,gregori2017evaluation} did.
\citet{dutta2017developing} also introduces a framework comparison based on the available features
and some psychological aspects of chatbot communication.
Unlike other references which either use or evaluate frameworks \citet{luis2015williams} and \citet{rasabocklisch2017} 
introduce the frameworks LUIS\cite{luis2015williams} and Rasa\cite{rasabocklisch2017}. 
They focus on the technology itself and don't build a chatbot with it nor do they compare frameworks.
In \citet{dutta2017developing, pharmacybot, PRZEGALINSKA2019785} actual chatbots for specific use cases have been implemented.



% ---------- END GOAL-ORIENTED SYSTEMS ---------- %

% ---------- DESIGN PSYCHO ASPECTS ---------- %
% folstad 2017 chatbots and the new world of HCI
\citet{folstad2017chatbots} investigates how users are communicating on the web and shows how 
the recent technological advances in the area of chatbots could influence the human-computer interaction in the future.
Lots of people are already using natural language as the main input method on the web through mobile messengers 
and social networks \cite{folstad2017chatbots}. 
At the moment the natural language conversation online is from human to human through a machine interface \cite{folstad2017chatbots}.
The people are using messenger platforms (machine-interface) to communicate with other people (human to human).
Some platforms already offer chatbot integration. 
At platforms like Twitter machine agents can already be integrated and communicate with human users \cite{folstad2017chatbots}. 
Chatbots are different from the classical software interfaces because they use natural language 
for communication instead of a classical GUI.
Currently, developers are focused on designing user interfaces but for chatbots the design focus 
lies on the conversation itself \cite{folstad2017chatbots}. 
Regular UI systems are designed by experts and improved by qualitative data which is rather sparse \cite{folstad2017chatbots}.
A big advantage of conversational interfaces is the massive amount of training data that is present all around the web \cite{folstad2017chatbots}. 
Chatbots all use messenger like interfaces independent from the problem.
Developers don't have to think about designing user interfaces for chatbots.
The developers need to switch to a goal-oriented view where the main focus 
lies on understanding what the user wants and how they can be served \cite{folstad2017chatbots}.
The state-of-the-art technology of conversational interfaces is Google Assistant \cite{folstad2017chatbots}.

% Brandtzaeg 2018 chatbots: Changing user needs and motivation
\citet{brandtzaeg2018chatbots} describes chatbots as a natural language interface using text or voice.
They are used to retrieve content or access a service through natural language conversation \cite{brandtzaeg2018chatbots}.
People are used to natural language communication because they spend lots of time on messenger platforms \cite{brandtzaeg2018chatbots}.
Because of this chatbot technologies become more and more popular \cite{brandtzaeg2018chatbots}.
Person ot person communication is like the open-ended chatbot approach.
It is difficult to design chatbots for open-ended conversations because users can start the conversation
in many ways \cite{brandtzaeg2018chatbots}.
The conversation can also develop in any direction.
The communication with a chatbot should feel natural.
Hence, it is important to balance the human and robot aspects of a chatbot \cite{brandtzaeg2018chatbots}.
If the chatbot is too human people might ask questions unrelated to the domain \cite{brandtzaeg2018chatbots}.
If it's too robot-like people might complain because the conversation feels unnatural \cite{brandtzaeg2018chatbots}.
Developers also need to think about how friendly a chatbot should be, how fast the bot should answer, if the bot should
have a gender, and how human-like the bot should be \cite{brandtzaeg2018chatbots}.
Successful chatbots inform the users what they have to expect and clarify that they are talking to a bot 
right from the start \cite{brandtzaeg2018chatbots}
Chatbots should inform users about what they can do and it can help to inform them what 
they can't do \cite{brandtzaeg2018chatbots}.
Conversational interfaces need to be improved based on the interactions with humans \cite{brandtzaeg2018chatbots}.
Three examples for chatbots are Microsofts Heston Bot for food, cooking opportunities and fashion, H\&Ms bot for shopping 
suggestions based on photos, and Ikeas shopping assistant bot \cite{brandtzaeg2018chatbots}.
\citet{brandtzaeg2018chatbots} focuse on psychological aspects of chatbot design and give recommendations
independent from the type of chatbot.

% humanize bots 2019
\citet{GO2019304} is about human-like bots and the 
effects they have on users and compares these bots with current chatbot systems.
The main functions of current online chatbots are interaction with users, address of concerns, and question answering \cite{GO2019304}.
These chatbots often lack humanness when communicating with users \cite{GO2019304}.
The impersonal nature of the conversation can be countered with a high level of message interactivity \cite{GO2019304}.
To increase the humanness of a chatbot three approaches can be used \cite{GO2019304}.
A chatbot can be given visual cues like human figures, identity cues like a human-associated name, or 
conversational cues by mimicking the human language \cite{GO2019304}.
Intorducing visual cues and a name to a chatbot is an easy task.
Conversational cues are hard to create since they require deep
understanding of human conversation and the context.
These three approaches can be used to make people believe that they are talking to a human or a bot.
The way a user interacts with a bot and the things a user expects change through this assumption \cite{sundar2016theoretical, GO2019304}.
If a user expects a human agent then the chatbot is more likely evaluated as human-like and the conversation 
feels more natural for users compared to when they expect a bot \cite{sundar2016theoretical}.
If the bot is identified as a human than the users automatically expect more from the conversation and the 
bot needs to be capable of more otherwise the users will be disappointed \cite{GO2019304}.
If the bot is identified as bot the users expect less from the conversation but the conversation 
is more likely evaluated as robotic or unnatural \cite{GO2019304}.
If on the other hand, the bot is falsely identified as a human being by a user the expectation of the user regarding 
interactivity rises \cite{GO2019304}.
This needs to be kept in mind when designing chatbots since the false identification as a human 
can lead to a huge decrease in user satisfaction and acceptance.
\citet{GO2019304} shows that the user assumption (bot or human) has a great impact 
on the chatbot evaluation and the feedback.

% ---------- EVAL ---------- %
Unlike the other references \citet{folstad2017chatbots, brandtzaeg2018chatbots, GO2019304}
focus on the psychological aspects of chatbots and the communication with machines in general.
\citet{GO2019304} is about the humanness of bots.
It's important to keep in mind how human a bot should be and how the user asumption influences 
the feedback.
If the bot appears to be human the expectations rise.
If the bot is presented as bot right from the start the expectations are in general lower.
\citet{brandtzaeg2018chatbots} also focuses on the psychological aspects and the humanness of bots.
The disadvantages of too human bots are listed with real life examples. 
Real chatbot conversations have shown that users ask unrealated and inappropropriate questions 
if a chatbot is too human. 
Hence, it's important to balance the robot and human aspects when designing a bot \cite{brandtzaeg2018chatbots}.
\citet{dutta2017developing} is about chatbot design aspects.
The important message of \citet{dutta2017developing} is that chatbot systems are completely different from regual systems from 
the design perspective.
Right now, developers focus on the design of user interfaces (UIs).
But chatbots all have the same UI requirements independent from the use case.
They either use messengers like UIs for text communication or no UI at all through speech.
This means that the the design focus of a chatbot isn't the UI.
It is the conversation.
Hence, the main design focus needs to shift from UI design to conversation design.
% ---------- END DESIGN PSYCHO ASPECTS ---------- %


% ---------- HERE ---------- %
% ---------- BASICS/CONCEPTS ---------- %
% deshpande 2017 survey
\citet{deshpande2017survey} shows the evolution of chatbots from the first chatbots to 
current state-of-the-art chatbots like Alexa and Siri.
The first chatbots created used pattern matching to respond to user inputs \cite{deshpande2017survey}.
Current technologies like Siri use NLP to respond to user inputs \cite{deshpande2017survey}.
With NLP the intent is identified and the question is analyzed to detect commands and actions \cite{deshpande2017survey}.
The basic workflow of a chatbot system starts with the user input \cite{deshpande2017survey}. 
Then the NLP engine extracts entities and detects the intent \cite{deshpande2017survey}.
Entities are used to retrieve the relevant data \cite{deshpande2017survey}.
After the relevant data has been retrieved a response is generated \cite{deshpande2017survey}.

% Building chatbots with Python
\citet{buiildChatbotsPython} explains the core concepts of chatbots, NLU, and NLP necessary for the development of chatbots.
A food order chatbot is built with Dialogflow and a horoscope bot is built with Rasa in \citet{buiildChatbotsPython}.
Chatbots are easy to use because it's as natural as talking to a human without the need for complex interfaces \cite{buiildChatbotsPython}.
Before starting to implement a bot the problem the bot should solve needs to be analyzed.
Chatbots can't do everything, in fact, they are most often designed for one specific task (task-oriented) \cite{buiildChatbotsPython}.
Some major questions need to be answered to determine if the problem is suited for the use of a chatbot \cite{buiildChatbotsPython}.
If the conversation is simple and consists of simple back-and-forth communication, and the task is 
highly repetitive and can be automated then the problem is suitable for a chatbot \cite{buiildChatbotsPython}.
It doesn't make much sense to implement a chatbot for a non-repetitive task which can't be automated \cite{buiildChatbotsPython}.
If the task is non repetitive then the development of the chatbot would take more 
time than the communication between the user and an employee.
A perfect example use-case for a chatbot is a FAQ web page \cite{buiildChatbotsPython}.
The answers are predefined, it's repetitive, it consists soly of questions and answers, and can be automated.
Some of the best chatbot and NLP frameworks are Rasa, LUIS, and Dialogflow \cite{buiildChatbotsPython}.

% geyer 2016 named entity recognition in 140 chars or less
\citet{geyer2016named} is about named entity recognition (NER) in microposts.
NER is the core concept used by chatbot tools to extract information from text.
The classic examples for named entities are people, locations, and organizations \cite{geyer2016named}.
\citet{geyer2016named} examines how good custom entities are recognized by the system.
The frequency of entities in the training data can influence the NER performance \cite{geyer2016named}. 
\citet{geyer2016named} checked the performance of entity recognition which can be done for chatbot frameworks in similar fashion.

% IEEE: rahman 2017 programming
\citet{rahman2017programming} gives a small overview of the concepts of Dialogflow and the general architecture of chatbot systems.
Commonly used cloud chatbot solutions are IBM Watson, Dialogflow, and Microsoft Bot \cite{rahman2017programming}.
In general, the chatbot architecture consists of intent classification, entity recognition, a response generator, 
and a response selector \cite{rahman2017programming}. 
The intent classification identifies the best matching intent for the users input.
The entity recognition module extracts the information/data from the user's message \cite{rahman2017programming}.
The response generator provides response candidates and the response selector chooses the best matching response \cite{rahman2017programming}.
Goal-oriented chatbots are most common in the business sector and help users to achieve tasks \cite{rahman2017programming}.
The tech giants provide chatbot frameworks for regular developers \cite{rahman2017programming}.
Google provides API.ai/Dialogflow, Microsoft has LUIS, IBM develops Watson, and Amazon provides Lex \cite{rahman2017programming}.
The key concepts of Dialogflow are intent, entity, and utterance.
Intents link the user's input to actions which are executed \cite{rahman2017programming}.

% ---------- EVAL ---------- %
\citet{deshpande2017survey,buiildChatbotsPython,geyer2016named,rahman2017programming} focus on the 
basics of chatbot frameworks and chatbot related concepts.
In \citet{deshpande2017survey, rahman2017programming} describe the general workflow of chatbot systems.
\citet{geyer2016named} is about NER and explains the core concept of entities.
Additionally, \citet{buiildChatbotsPython} also explains which problems are 
suitable for a chatbot in general.
% ---------- END BASICS/CONCEPTS ---------- %

% ---------- EXAMPLES ---------- %

% ---------- END EXAMPLES ---------- %


% ISO 25010
ISO/IEC 25010 defines quality characteristics for the evaluation of software systems \cite{iso25010}.
Functional suitability is split into functional completeness, correctness and appropriateness \cite{iso25010}.
There are also performance efficiency metrics to check the time behavior of a system with the throughput time \cite{iso25010}.
The interoperability of systems can be verified by checking the exchanged information \cite{iso25010}.
Usability can be checked through learnability which determines how easy it is to learn to use the product \cite{iso25010}.
Security can be checked by checking for unauthorized access problems \cite{iso25010}.
A part of portability is how easy it is to install a system \cite{iso25010}.
ISO/IEC 25010 provides evaluation criteria for software systems which can be used to evaluate chatbot systems.


% my conclusion
The ML and NLP capabilities of the frameworks are tested in some papers but the actual frameworks are 
not really compared. 
Most papers focus on the NLP capabilities and not on other aspects like usability, simplicity or the comparison of the 
functionality of the different frameworks.
Papers like \citet{braunEvaluatingNLU} show that the differences of chatbot frameworks is small when it comes to NLU performance.
Hence, it doesn't make a huge difference which framework is chosen when the focus lies on NLU.
The NLU capabilities of the framework are important in general but to give a development recommendation other aspects
like the available features and the simplicity of a framework are more important.

% ---------------------------- HERE ---------------------------- %


